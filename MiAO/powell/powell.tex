\documentclass[a4paper, 12pt]{article}
\usepackage[utf8]{inputenc}
\usepackage{geometry}
\usepackage{polski}
\usepackage{graphicx}
\usepackage{float}
\usepackage{etoolbox,refcount}
\usepackage{multicol}
\usepackage{fancyhdr}
\usepackage{listings}
\usepackage{amsmath}
\usepackage{tabularx}
\usepackage{svg}

\newgeometry{left=2.5cm, right=2.5cm, bottom=2.5cm, top=2.5cm}


\begin{document}
	\noindent
	\resizebox{\textwidth}{!}{
		\begin{tabular}{|c|c|c|c|}
			\hline 
			\multicolumn{4}{|l|}{Optymalizacja na kierunku} \\ 
			\hline 
			Adrian Jałoszewski & 29 III 2017 & Środa 14:00 &  \\ 
			\hline 
		\end{tabular}
	} 
	\section{Cel ćwiczenia}
		Celem ćwiczenia było zapoznanie się z obydwiema metodami Powella dla przypadku funkcjonału kwadratowego oraz doliny bananowej Rossenbrocka.
	\section{Przebieg ćwiczenia}
		\subsection{Zadanie 1}
			Zadanie pierwsze polegało na minimalizacji funkcjonału kwadratowego postaci:
			$$
				Q(x) = x^T A x + b^T x + c
			$$
			Minimalizacja przebiegała obiema metodami Powella dla następujących parametrów:
			\begin{itemize}
				\item[] $A = \begin{bmatrix}
				8 & -2 \\ 
				-2 & 5
				\end{bmatrix}$
				\item[] $b = \begin{bmatrix}
				3 \\ 
				4
				\end{bmatrix}$ 
				\item[] $c = 10$
			\end{itemize}
			Punktem początkowym był punkt $x = [7, 4]^T$
			\begin{figure}[H]
				\centering
				\def \svgwidth{0.7\columnwidth}
				\input{1_1.pdf_tex}
				\caption{Pierwsza metoda Powella}
			\end{figure}\noindent
			\begin{figure}[H]
				\centering
				\def \svgwidth{0.7\columnwidth}
				\input{1_2.pdf_tex}
				\caption{Druga metoda Powella}
			\end{figure}\noindent
		\subsection{Zadanie 2}
			Dolina bananowa Rossenbrocka jest dana równaniem:
			$$
				Q(x_1, x_2) = 100 (x_2 - x_1^2)^2 + (1 - x_1)^2
			$$
			\begin{figure}[H]
				\centering
				\def \svgwidth{0.7\columnwidth}
				\input{2_1.pdf_tex}
				\caption{Pierwsza metoda Powella}
			\end{figure}\noindent
			\begin{figure}[H]
				\centering
				\def \svgwidth{0.7\columnwidth}
				\input{2_2.pdf_tex}
				\caption{Druga metoda Powella}
			\end{figure}\noindent
		\subsection{Zadanie 3}
			Należy zbadać wartości funkcji celu dla maksymalnie 100 iteracji metod Powella. Funkcja celu była dana jako:
			$$
				\begin{aligned}
					Q(x) = 100(x_1^2-x_2) ^ 2 + (1 - x_1)^2 + 90 (x_3^3 - x_4)^2
					+ (1 - x_3)^2 \\
					+ 10.1 [(x_2 - 1)^2 + (x_4-1)^2] + 19.8 (x_2-1)(x_4-1)
				\end{aligned}
			$$
			\begin{figure}[H]
				\centering
				\def \svgwidth{0.7\columnwidth}
				\input{3_1.pdf_tex}
				\caption{Pierwsza metoda Powella}
			\end{figure}\noindent
			\begin{figure}[H]
				\centering
				\def \svgwidth{0.7\columnwidth}
				\input{3_2.pdf_tex}
				\caption{Druga metoda Powella}
			\end{figure}\noindent
	\section{Wnioski z wykonanego ćwiczenia}
		Wykresy w zadaniu pierwszym pokazały, że zarówno pierwsza i druga metoda Powella są szybko zbieżne dla funkcjonałów kwadratowych, gdyż zaledwie w dwóch krokach wyznaczyły położenie minimum. Ponieważ obydwie metody posiadają zbieżność drugiego rzędu, to trudno tutaj powiedzieć o przypadku dla którego występuje brak zbieżności. Można jednak powiedzieć o przypadku powolnej zbieżności jeżeli wybrana początkowa baza będzie bliska zdegenerowanej (na podstawie skryptu).
		\\ \\
		W zadaniu drugim druga metoda okazała się zdecydowanie wolniej zbieżna od pierwszej. 
		\\ \\
		W zadaniu trzecim druga metoda Powella wykonała wszystkie iteracje, pierwsza zakończyła się po pięciu iteracjach. W przypadku drugiej metody Powella funkcja celu maleje wolniej, jest to powiązane z tym, że pierwsze dwa człony to dolina bananowa Rossenbrocka (potwierdzenie obserwacji z zadania drugiego).
\end{document}
