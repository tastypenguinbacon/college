\documentclass[a4paper, 12pt]{article}
\usepackage[utf8]{inputenc}
\usepackage{geometry}
\usepackage{polski}
\usepackage{graphicx}
\usepackage{float}
\usepackage{etoolbox,refcount}
\usepackage{multicol}
\usepackage{fancyhdr}
\usepackage{listings}
\usepackage{amsmath}
\usepackage{tabularx}
\usepackage{svg}

\newgeometry{left=2.5cm, right=2.5cm, bottom=2.5cm, top=2.5cm}


\begin{document}
	\noindent
	\resizebox{\textwidth}{!}{
		\begin{tabular}{|c|c|c|c|}
			\hline 
			\multicolumn{4}{|l|}{Optymalizacja wielokryterialna} \\ 
			\hline 
			Adrian Jałoszewski & 24 V 2017 & Środa 14:00 &  \\ 
			\hline 
		\end{tabular}
	} 
	\section{Cel ćwiczenia}
        Celem ćwiczenia jest zapoznanie się zastosowaniami optymalizacji wielokryterialnej na
        przykładzie optymalizacji parametrów dwójnika elektrycznego, regulatora proporcjonalnego 
        z inercją oraz dla przypadku, gdzie mamy do czynienia z wysokim kosztem pozyskania danych.
	\section{Przebieg ćwiczenia}
        \subsection{Zadanie 1}
            Zadanie polega na optymalizacji dwójnika elektrycznego pod względem sprawności oraz 
            wydzielanej mocy na rezystancji $R_a$. Rezystancja $R_i$ jest ograniczona -- jej wartość 
            mieści się w przedziale $[3, 5]$. Równanie napięciowe Kirchhoffa dla tego układu:
            $$
                E - iR_i - iR_a = 0
            $$
            Wynika z tego, że prąd płynący przez obwód jest zadany wzorem:
            $$
                i = \frac{E}{R_i + R_a}
            $$
            Moc wydzielana na rezystancji jest dana zależnością:
            $$
                P_a = R_a i^2 = \frac{R_a}{(R_i + R_a)^2} E ^ 2
            $$
            Ponieważ wartość ta jest zależna od wartości napięcia źródła zasilania, lepiej jest zastąpić
            te kryterium przy pomocy kryterium równoważnego:
            $$
                \mu = \frac{P_a}{E^2} = \frac{R_a}{(R_i + R_a)^2}
            $$
            Sprawność jest zadana jako stosunek mocy wydzielanej na rezystancji $R_a$ do mocy
            wydzielanej na obydwu rezystancjach.
            $$
                \eta = \frac{P_a}{P_a + P_i} = \frac{R_ai^2}{R_ai^2 + R_ii^2} = \frac{R_a}{R_a + R_i}
            $$
            \begin{figure}[H]
                \centering
                \def \svgwidth{0.7\columnwidth}
                \input{power.pdf_tex}
                \caption{Moc wydzielana na rezystancji $R_a$ w stosunku do kwadratu napięcia}
            \end{figure}\noindent
            
            \begin{figure}[H]
                \centering
                \def \svgwidth{0.7\columnwidth}
                \input{efficiency.pdf_tex}
                \caption{Sprawność układu}
            \end{figure}\noindent
            Można zauważyć, że człon $\frac{R_a}{R_a + R_i}$ występuje w obydwu wskaźnikach jakości.
            Można w ten sposób uzależnić wskaźniki jakości od siebie i jednej z wartości. Otrzymuje
            się w ten sposób zależność mocy wydzielonej jako:
            \begin{equation}
                \mu = \frac{1}{R_i} (1 - \eta) \eta
                \label{eq:funs}
            \end{equation}
            Jest to rodzina paraboli o wierzchołkach leżących na prostej $\eta = \frac{1}{2}$, 
            z parametrem $R_i$, który określa jego położenie. Istnieje również prostsza zależność 
            $\mu = \frac{1}{R_a}\eta^2$, jednak jej wadą jest to, że dla każdej paraboli istnieje
            parabola lepsza -- taka o mniejszym $R_a$, a ponieważ tylko 0 jest ograniczeniem, to 
            taka parabola nie istnieje. Dlatego do wyznaczenia zbioru kompromisów lepiej jest
            wziąć zależność \ref{eq:funs}.
            \begin{figure}[H]
                \centering
                \def \svgwidth{0.7\columnwidth}
                \input{contour_overlay.pdf_tex}
                \caption{Rodzina zbiorów rozwiązań kompromisowych}
            \end{figure}\noindent
        \subsection{Zadanie 2}
            Zadanie polega na dobraniu tak parametrów układu regulacji aby zminimalizować uchyb
            statyczny, równocześnie minimalizując przeregulowanie. Zakładam, że zakłócenie jest
            zerowe. Transmitancja układu otwartego jest wtedy dana jako:
            $$
                G_o(s) = \frac{K}{T_oT_rs^2 + (T_o + T_r)s + 1}
            $$
            Transmitancja układu zamkniętego jest wtedy dana jako:
            $$
                G(s) = \frac{K}{T_oT_rs^2 + (T_o + T_r)s + (K + 1)}
            $$
            Aby można było rozważać ten układ w kategoriach stabilności, musi on być stabilny. Ponieważ
            parametr odpowiadający za tłumienie w układzie jest dodatni, to układ jest stabilny
            asymptotycznie pod warunkiem, że wyraz wolny jest dodatni $K > -1$.
            \\ \\
            Uchyb statyczny jest dany wzorem:
            $$
                e_s =  \lim_{t \to \infty} e(t) = \lim_{s \to 0} sE(s)
            $$
            Gdzie $E(s)$ jest dane wzorem:
            $$
                E(s) = W(s) - Y(s) = W(s) - G_o(s) E(s)
            $$
            $$
                E(s) = \frac{W(s)}{1 + G_o(s)}
            $$
            Ponieważ jest to układ liniowy, a rozważanie samego uchybu statycznego jest mało miarodajne
            można przyjąć dowolną wartość dla sygnału skokowego i rozważać względną wartość uchybu:
            $$
                e_w(s) = \frac{\frac{1}{s}}{(1 + \frac{K}{T_oT_rs^2 + (T_o + T_r)s + 1})}
            $$
            $$
                e_{ws} = \lim_{s \to 0} s \frac{\frac{1}{s}}{(1 + \frac{K}{T_oT_rs^2 + 
                    (T_o + T_r)s + 1})} = \frac{1}{1 + K}
            $$
            Do wyznaczenia względnego przeregulowania można posłużyć się wzorem:
            \begin{equation}
                PO = \exp\left(
                    - \frac{\zeta \pi}{\sqrt{1 - \zeta^2}}
                \right)
                \label{eq:overshoot}
            \end{equation}
            $\zeta$ jest tutaj współczynnikiem tłumienia, który można odczytać z wzoru na transmitancję
            układu, gdy jest postaci:
            $$
                G(s) = \frac{K}{T^2s^2 + 2\zeta Ts + 1}
            $$
            Transmitancję całego układu należy jednak najpierw sprowadzić do tej formy:
            $$
                G(s) = \frac{K}{T_rT_os^2 + (T_r + T_o)s + K + 1} =
                    \frac{\frac{K}{K+1}}{\frac{T_rT_o}{K + 1}s^2 + \frac{T_r + T_o} {K + 1} s + 1}
            $$
            Można na podstawie tego odczytać, że:
            $$
                \begin{aligned}
                    T =& \sqrt{\frac{T_rT_o}{K + 1}} \\
                    \zeta =& \frac{T_r + T_o}{2 T (K + 1)}
                \end{aligned}
            $$
            Równanie \ref{eq:overshoot}  posiada ograniczenie, że $\zeta < 1$, gdy współczynnik 
            ten jest większy należy przyjąć przeregulowanie jako zerowe, gdyż zjawisko te wtedy 
            nie występuje. Podstawiając do równanania na współczynnik $\zeta$ wartość współczynnika
            $T$ otrzymuje się:
            \begin{equation}
                \zeta = \frac{T_0 + T_r}{2} \frac{1}{\sqrt{T_rT_o}} \frac{1}{K + 1}
                \label{eq:zeta}
            \end{equation}
            Aby pozbycć się $T_o$ z zależności \ref{eq:zeta} można zmienić bez straty zmienić zmienną
            decyzyjną $T_o$ na zmieną decyzyjną $x$ będącą ilorazem średniej arytmetycznej oraz 
            geometrycznej. Rozwiązanie te ma zaletę uniezależnienia rozważań od wartości parametru
            $T_o$, wadą tego rozwiązania jest natomiast, że nie odwzorowanie to nie jest injekcją.
            $$
                x = \frac{T_o + T_r}{2\sqrt{T_rT_o}} = \frac{1}{2} 
                \left(
                    \sqrt{\frac{T_o}{T_r}} + \sqrt{\frac{T_r}{T_o}}
                \right)
            $$
            Konieczne jest wyznaczenie pochodnej $\frac{\mathrm{d}x}{\mathrm{d}T_r}$ w celu ustalenia
            zachowania tego odwzorowania.
            $$
                \frac{\mathrm{d}x}{\mathrm{d}T_r} = \frac{1}{4} \left(
                    \frac{T_o}{T_r^{3/2}} +
                    \frac{1}{T_o T_r^{1/2}}
                \right)
            $$
            Z wartości pochodnej wynika, że:
            \begin{itemize}
                \item Dla $T_r > T_o$ odwzorowanie jest rosnące
                \item Dla $T_r < T_o$ odwzorowanie jest malejące
                \item Dla $T_r = T_o$ odwzorowanie osiąga minium równe 1
            \end{itemize}
            Dla wartości $T_r \to \infty$ oraz $T_r \to 0$ wartość odwzorowania zmierza do 
            nieskończoności.  Z tego oraz z wartości pochodnych funkcji wynika, że dla każdej 
            wartości $T_r$ za wyjątkiem $T_r = T_o$ odwzorowanie te posiada przyjmuje tę samą 
            wartość dla dwóch różnych argumentów, a jego przeciwdziedzina należy do $[1, +\infty)$.
            \\ \\
            Dokonując zamiany wartości otrzymuję następujący wzór na współczynnik $\zeta$:
            \begin{equation}
                \zeta = \frac{x}{K+1}
            \end{equation}
            Po wprowadzeniu tej zmiany równanie na wartość uchybu statycznego nie ulega zmianom. Zamiana
            ta ma tylko wpływ na postać równania przeregulowania, które po podstawieniu ma teraz postać:
            \begin{equation}
                PO = \exp\left(
                    - \frac{\frac{x}{K+1} \pi}{\sqrt{1 - \left(\frac{x}{K+1}\right)^2}}
                \right)
                \label{eq:overshoot2}
            \end{equation}
            Z ograniczenia $\zeta < 1$ wynika ograniczenie $x < K + 1$. Jak to nie jest spełnione, 
            to w układzie nie występuje zjawisko przeregulowania (na wykresie nad białą linią).
            \begin{figure}[H]
                \centering
                \def \svgwidth{0.7\columnwidth}
                \input{overshoot.pdf_tex}
                \caption{Wartość przesterowania}
            \end{figure}\noindent
            Ponieważ wartości uchybu statycznego są zależne tylko od jednej zmiennej można je
            przedstawić na dwuwymiarowym wykresie.
            \begin{figure}[H]
                \centering
                \def \svgwidth{0.7\columnwidth}
                \input{static2d.pdf_tex}
                \caption{Wartość uchybu statycznego}
            \end{figure}\noindent
            Zależność kryteriów od siebie jest dana jako:
            $$
                PO = \exp\left(\frac{-\pi}{\sqrt{(xe_{ws})^{-2} - 1}}\right)
            $$
            \begin{figure}[H]
                \centering
                \def \svgwidth{0.7\columnwidth}
                \input{overshoot_static_overlay.pdf_tex}
                \caption{Zależności między kryteriami}
            \end{figure}\noindent
            Na podstawie tego można wywnioskować, że najlepszym przypadkiem jest ten dla którego nie ma 
            przeregulowania, a wartość uchybu jest zerowa $K \to \infty$. Zbiór kompromisów dla tego
            przypadku składa się z jednego punkut, gdyż dla pozostałych zawsze znajdzie się lepszy
            (z większym współczynnikiem $x$ -- większa rozbieżność między stałymi czasowymi. 
            \begin{figure}[H]
                \centering
                \def \svgwidth{0.7\columnwidth}
                \input{badumpts.pdf_tex}
                \caption{Przebieg odpowiedzi skokowej dla dobranych parametrów}
            \end{figure}\noindent
        \subsection{Zadanie 3}
            Zadanie polega na maksymalizacji współczynnika skuteczności leku przy równoczesnej
            minimalizacji jego nieskuteczności. Wartości są ograniczone poprzez $Q_1 + Q_2 \leq 100\%$
            \begin{figure}[H]
                \centering
                \def \svgwidth{0.7\columnwidth}
                \input{scatter.pdf_tex}
                \caption{Porównanie kryteriów}
            \end{figure}\noindent
            Do rozważań skuteczności leku warto wziąć pod uwagę procent badanych, który doświadczył
            jakiejkolwiek poprawy zdrowia dany jako $Q_3 = 100\% - Q_2$, do którego należy jednak 
            podejść krytycznie ze względu na możliwość wystąpienia efektu placebo.
            \begin{figure}[H]
                \centering
                \def \svgwidth{0.7\columnwidth}
                \input{only_working.pdf_tex}
                \caption{Porównanie skuteczności}
            \end{figure}\noindent
            Jest to zdecydowanie lepszy przypadek w rozważaniu, gdyż biorąc w rozważania wskaźniki $Q_1$
            oraz $Q_3$ mamy do czynienia z zagadnieniem maksymalizacji dwóch wskaźników. Ograniczenia są
            wtedy nastepujące:
            $$
                \begin{aligned}
                    Q_1 + Q_2 \leq& 100\% \\
                    Q_1 + 100\% - Q_3 \leq& 100\% \\
                    Q_1 \leq& Q_3
                \end{aligned}
            $$
            \begin{figure}[H]
                \centering
                \def \svgwidth{0.7\columnwidth}
                \input{together.pdf_tex}
                \caption{Porównanie skuteczności, nieskuteczności oraz braku nieskuteczności}
            \end{figure}\noindent
            Zbiorem kompromisow w tym przypadku są próbki 1, 3, 5, gdyż pozostałe przypadki spełniają 
            jedno lub więcej kryteriów gorzej niż pozostałe.
    \section{Wnioski i obserwacje z wykonanego ćwiczenia}
        \subsection{Zadanie 1}
            Zbiór kompromisów jest tutaj fragmentem paraboli znajdującym się po jednej ze stron 
            wierzchołka. Zależność pomiędzy poszczególnymi kryteriami można tutaj wyznaczyć na 
            dwa sposoby, jednak ze względu na brak ograniczeń na jedną ze zmiennych decyzyjnych
            tylko przy pomocy jednej z nich dało się wyznaczyć zbiór kompromisów.
        \subsection{Zadanie 2}
            W tym przykładzie bardzo dobrze można uwidocznić jak zmiana zmiennych decyzyjnych może 
            pomóc uogólnić rozważania. Warto zauważyć, że w tym przypadku optymalizacji jest wziętych
            pod uwagę za mało kryteriów. Przykładowo -- rozwiązaniami, które posiadają tę samą wartość
            funkcji celu jest zarówno przypadek tłumienia krytycznego jak i przypadek, który się zbliży
            do wartości ustalonej na dzień przed zużyciem wodoru przez Słońce. Istnieje tutaj tylko
            jeden punkt w zbiorze kompromisów, gdyż dla każdego innego przypadku ze względu na brak
            ograniczeń we wzmocnieniu da się znaleźć przypadek lepszy -- zwiększenie rozbieżności 
            pomiędzy stałymi czasowymi skutkuje w przesunięciu wykresu bliżej osi. Jedynie jeden 
            punkt wspólny dla wszystkich wykresów równocześnie minimalizuje jeden ze wskaźników lepiej
            niż pozostałe. 
        \subsection{Zadanie 3}
            Podobnie jak w poprzednim zadaniu, tak i tutaj zamiana zmiennych decyzyjnych pozwala
            uprościć rozważania, jednak w tym przypadku upaszczanie polega na zamianie zagadnienia
            równoczesnej minimalizacji i maksymalizacji wskaźników na maksymalizację obydwu. Warto
            zwrócić tutaj uwagę na to, że wariantem leku, który by wszedł na rynek byłby wariant 
            trzeci, gdyż jest on najskuteczniejszy, jednak biorąc pod uwagę możliwośc niezadowolenia
            konsumentów z niedziałającego leku warto również wziąć pod uwagę jego nieskuteczność 
            lub też jakąkolwiek skuteczność i na podstawie nich skonstruować funkcję celu.
\end{document}
