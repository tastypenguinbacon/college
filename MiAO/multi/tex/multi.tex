\documentclass[a4paper, 12pt]{article}
\usepackage[utf8]{inputenc}
\usepackage{geometry}
\usepackage{polski}
\usepackage{graphicx}
\usepackage{float}
\usepackage{etoolbox,refcount}
\usepackage{multicol}
\usepackage{fancyhdr}
\usepackage{listings}
\usepackage{amsmath}
\usepackage{tabularx}
\usepackage{svg}

\newgeometry{left=2.5cm, right=2.5cm, bottom=2.5cm, top=2.5cm}


\begin{document}
	\noindent
	\resizebox{\textwidth}{!}{
		\begin{tabular}{|c|c|c|c|}
			\hline 
			\multicolumn{4}{|l|}{Optymalizacja wielokryterialna} \\ 
			\hline 
			Adrian Jałoszewski & 24 V 2017 & Środa 14:00 &  \\ 
			\hline 
		\end{tabular}
	} 
	\section{Cel ćwiczenia}
        Celem ćwiczenia jest zapoznanie się zastosowaniami optymalizacji wielokryterialnej na
        przykładzie optymalizacji parametrów dwójnika elektrycznego, regulatora proporcjonalnego 
        z inercją oraz dla przypadku, gdzie mamy do czynienia z wysokim kosztem pozyskania danych.
	\section{Przebieg ćwiczenia}
        \subsection{Zadanie 1}
            Zadanie polega na optymalizacji dwójnika elektrycznego pod względem sprawności oraz 
            wydzielanej mocy na rezystancji $R_a$. Równanie napięciowe Kirchhoffa dla tego układu:
            $$
                E - iR_i - iR_a = 0
            $$
            Wynika z tego, że prąd płynący przez obwód jest zadany wzorem:
            $$
                i = \frac{E}{R_i + R_a}
            $$
            Moc wydzielana na rezystancji jest dana zależnością:
            $$
                P_a = R_a i^2 = \frac{R_a}{(R_i + R_a)^2} E ^ 2
            $$
            Sprawność jest zadana jako stosunek mocy wydzielanej na rezystancji $R_a$ do mocy
            wydzielanej na obydwu rezystancjach.
            $$
                \mu = \frac{P_a}{P_a + P_i} = \frac{R_ai^2}{R_ai^2 + R_ii^2} = \frac{R_a}{R_a + R_i}
            $$
            \begin{figure}[H]
                \centering
                \def \svgwidth{0.7\columnwidth}
                \input{power.pdf_tex}
                \caption{Moc wydzielana na rezystancji $R_a$}
            \end{figure}\noindent
            
            \begin{figure}[H]
                \centering
                \def \svgwidth{0.7\columnwidth}
                \input{efficiency.pdf_tex}
                \caption{Sprawność układu}
            \end{figure}\noindent
            \begin{figure}[H]
                \centering
                \def \svgwidth{0.7\columnwidth}
                \input{contour_overlay.pdf_tex}
                \caption{Zbiór rozwiązań kompromisowych}
            \end{figure}\noindent
        \subsection{Zadanie 2}
            Zadanie polega na dobraniu tak parametrów układu regulacji aby zminimalizować uchyb
            statyczny, równocześnie minimalizując przeregulowanie. Zakładam, że zakłócenie jest
            zerowe. Można wtedy układ sprowadzić do natępującej formy:
            \begin{figure}[H]
                \centering
                \def \svgwidth{0.7\columnwidth}
                \input{changed_system.pdf_tex}
                \caption{Zbiór rozwiązań kompromisowych}
            \end{figure}\noindent
            Transmitancja układu otwartego jest wtedy dana jako:
            $$
                G_o(s) = \frac{K}{T_oT_rs^2 + (T_o + T_r)s + 1}
            $$
            Transmitancja układu zamkniętego jest wtedy dana jako:
            $$
                G(s) = \frac{K}{T_oT_rs^2 + (T_o + T_r)s + (K + 1)}
            $$
            Aby można było rozważać ten układ w kategoriach stabilności, musi on być stabilny. Ponieważ
            parametr odpowiadający za tłumienie w układzie jest dodatni, to układ jest stabilny
            asymptotycznie pod warunkiem, że wyraz wolny jest dodatni $K > -1$.
            \\ \\
            Uchyb statyczny jest dany wzorem:
            $$
                e_s =  \lim_{t \to \infty} e(t) = \lim_{s \to 0} sE(s)
            $$
            Gdzie $E(s)$ jest dane wzorem:
            $$
                E(s) = W(s) - Y(s) = W(s) - G_o(s) E(s)
            $$
            $$
                E(s) = \frac{W(s)}{1 + G_o(s)}
            $$
            Ponieważ jest to układ liniowy, a rozważanie samego uchybu statycznego jest mało miarodajne
            można przyjąć dowolną wartość dla sygnału skokowego i rozważać względną wartość uchybu:
            $$
                e_w(s) = \frac{\frac{1}{s}}{(1 + \frac{K}{T_oT_rs^2 + (T_o + T_r)s + 1})}
            $$
            $$
                e_{ws} = \lim_{s \to 0} s \frac{\frac{1}{s}}{(1 + \frac{K}{T_oT_rs^2 + 
                    (T_o + T_r)s + 1})} = \frac{1}{1 + K}
            $$
            Do wyznaczenia względnego przeregulowania można posłużyć się wzorem:
            $$
                PO = \exp\left(
                    - \frac{\zeta \pi}{\sqrt{1 - \zeta^2}}
                \right)
            $$
            $\zeta$ jest tutaj współczynnikiem tłumienia, który można odczytać z wzoru na transmitancję
            układu, gdy jest postaci:
            $$
                G(s) = \frac{K}{T^2s^2 + 2\zeta Ts + 1}
            $$
        \subsection{Zadanie 3}
    \section{Wnioski i obserwacje z wykonanego ćwiczenia}
        \subsection{Zadanie 1}
        \subsection{Zadanie 2}
        \subsection{Zadanie 3}
\end{document}
