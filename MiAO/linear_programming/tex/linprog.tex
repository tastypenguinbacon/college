\documentclass[a4paper, 12pt]{article}
\usepackage[utf8]{inputenc}
\usepackage{geometry}
\usepackage{polski}
\usepackage{graphicx}
\usepackage{float}
\usepackage{etoolbox,refcount}
\usepackage{multicol}
\usepackage{fancyhdr}
\usepackage{listings}
\usepackage{amsmath}
\usepackage{tabularx}
\usepackage{svg}

\newgeometry{left=2.5cm, right=2.5cm, bottom=2.5cm, top=2.5cm}


\begin{document}
	\noindent
	\resizebox{\textwidth}{!}{
		\begin{tabular}{|c|c|c|c|}
			\hline 
			\multicolumn{4}{|l|}{Programowanie liniowe} \\ 
			\hline 
			Adrian Jałoszewski & 26 IV 2017 & Środa 14:00 &  \\ 
			\hline 
		\end{tabular}
	} 
	\section{Cel ćwiczenia}
        Celem ćwiczenia jest dokonanie optymalizacji dwóch funkcji celu na zadanych ograniczeniach.
        Jednym z przypadków jest zadanie maksymalizacji zysków przy ograniczeniach wynikających z
        możliwości przerobowych przedsięcbiorstwa. Drugie zadanie polega na takim doborze rezystancji 
        aby mając ustalone napiecia na poszczególnych rzezystancjach i ograniczenia na przepływający 
        przez nie prąd zminimalizować zużycie mocy przez układ.
	\section{Przebieg ćwiczenia}
		\subsection{Zadanie 1}
            Ćwiczenie należy rozpocząć od przedstawienia zagadnienia w postaci zagadnienia 
            programowania liniowego. Oznaczmy $x_A$, $x_B$, $x_C$ jako kolejno ilość wytworzonego 
            danego typu. Wtedy zgodnie z zadanymi ograniczeniami zachodzi:
            $$
                \begin{aligned}
                    0.3 \cdot x_A + 0.5 \cdot x_B + 0.4 \cdot x_C \leq & 1800 \\
                    0.1 \cdot x_A + 0.08 \cdot x_B + 0.12 \cdot x_C \leq & 500 \\
                    0.06 \cdot x_A + 0.04 \cdot x_B + 0.05 \cdot x_C \leq & 200 \\
                    x_A, x_B, x_C \geq & 0
                \end{aligned}
            $$
            Funkcja celu jest zadana w tym przypadku jako:
            $$
                f(x_A, x_B, x_C) = 400 \cdot x_A + 300 \cdot x_B + 200 \cdot x_C
            $$
            \begin{figure}[H]
                \centering
                \def \svgwidth{0.7\columnwidth}
                \input{first.pdf_tex}
                \caption{Graficzne przedstawienie rozwiązania}
            \end{figure}\noindent
            Podstawa  wielościanu będącego ograniczeniami w zadaniu jest następująca:
            \begin{figure}[H]
                \centering
                \def \svgwidth{0.7\columnwidth}
                \input{bottom.pdf_tex}
                \caption{Podstawa graniastosłupe}
            \end{figure}\noindent
            Kropki zaznaczają znalezione maksimum, które się znajduje w przybliżeniu w punkcie
            $(x_A, x_B, x_C) = (1556, 2667, 0)$, wynosi ono 1 422 222.
        \subsection{Zadanie 2} 
            Najpierw należy sprowadzić zadany układ do postaci kanonicznej zadania programowania
            liniowego. Jest to zadanie minimalizacji z ograniczeniami na to aby wartości prądów były
            nieujemne i ograniczenia górne na ich kombinacje liniowe.
            \\ \\
            Minimalizowana funkcja celu dana jest wzorem na sumę mocy wydzielanych na poszczególnych
            rezystorach:
            $$
                f(I_1, I_2, I_3, I_4, I_5) = 6 I_1 + 10 I_2 + 4 I_3 + 7 I_4 + 3 I_5
            $$
            Każde z ograniczeń postaci:
            $$
                I_i - D_i \leq I_i \leq I_i + D_i
            $$
            można sprowadzić do następującej pary ograniczeń:
            $$
                \begin{aligned}
                    I_i \leq I_i + D_i \\
                    - I_i \leq D_i - I_i
                \end{aligned}
            $$
            Aby układ był realizowalny fizycznie muszą być spełnione prawa Kirchhoffa. Ponieważ
            napięciowe prawa są już narzucone wystarczy wypisać prawa prądowe dla dwóch węzłów:
            $$
                \begin{aligned} 
                    I_1 =& I_3 + I_4 \\
                    I_2 + I_3 =& I_5
                \end{aligned}
            $$
            Obydwie te równości można zapisać w postaci czterech nierównośći: 
            $$
                \begin{aligned}
                    I_1 - I_3 - I_4 \leq & 0 \\
                    -I_1 + I_3 + I_4 \leq & 0 \\
                    I_2 + I_3 - I_5 \leq & 0 \\
                    -I_2 - I_3 + I_5 \leq & 0
                \end{aligned}
            $$
            Mimo tego, że jest to układ mostka -- nie trzeba rozważać przypadku kiedy prąd przez 
            rezystor $R_3$ płynie w drugą stronę -- kierunek przepływu prądu jest już narzucony
            przez ograniczenia narzucone w treści zadania.
            \\ \\
            W ten sposób wszystkie ograniczenia można przedstawić w następującej postaci:
            $$
                \begin{bmatrix}
                    1 & 0 & 0 & 0 & 0 \\
                    -1 & 0 & 0 & 0 & 0 \\

                    0 & 1 & 0 & 0 & 0 \\
                    0 & -1 & 0 & 0 & 0 \\

                    0 & 0 & 1 & 0 & 0 \\
                    0 & 0 & -1 & 0 & 0 \\

                    0 & 0 & 0 & 1 & 0 \\
                    0 & 0 & 0 & -1 & 0 \\

                    0 & 0 & 0 & 0 & 1 \\
                    0 & 0 & 0 & 0 & -1 \\

                    1 & 0 & -1 & -1 & 0 \\
                    -1 & 0 & 1 & 1 & 0 \\

                    0 & 1 & 1 & 0 & -1 \\
                    0 & -1 & -1 & 0 & 1
                \end{bmatrix} 
                \cdot
                \begin{bmatrix}
                    I_1 \\
                    I_2 \\
                    I_3 \\
                    I_4 \\
                    I_5
                \end{bmatrix}
                \leq 
                \begin{bmatrix}
                    5\\
                    -3\\

                    3\\
                    -1\\

                    3\\
                    -1\\

                    3\\
                    -1\\

                    4\\
                    -4\\
                    
                    0\\
                    0\\
                    
                    0\\
                    0
                \end{bmatrix}
            $$
            Oraz warunki o nieujemności prądów:
            $$
                I_1, I_2, I_3, I_4, I_5 \geq 0
            $$
            Wyznaczenie wartości prądów jest w zupełności wystarczające do wyznaczenia wartości
            rezystancji, gdyż są podane również napięcia. Gdyby rozwiązywać te zadanie wykorzystując
            rezystancje jako zmienne decyzyjne funkcja celu nie byłaby funkcją liniową, a 
            ograniczenia uległyby komplikacjom.
            \\
            \\
            Minimum o wartości 65 mW otrzymuje się dla prądów o wartości $(I_1, I_2, I_3, I_4, I_5)
            = (4, 1, 3, 1, 4)$ miliamperów, czyli dla rezystancji o wartościach: 
            $$
                (R_1, R_2, R_3, R_4, R_5) = 
                \mathrm{(1.5k\Omega, 10k\Omega, 1.333k\Omega, 7k\Omega, 0.75k\Omega)}
            $$
            Do rozważań ilości rozwiązań należy wziąć pod uwagę fakt, że warunki z zadania składają
            się na układ trzech równań i czterech nierówności. Z równań wynika, że:
            $$
                \begin{aligned}
                    I_5 =& 4 \\
                    I_2 =& 4 - I_3 \\
                    I_1 =& I_3 + I_4
                \end{aligned}
            $$
            Podstawiając te trzy wartości do czterech nierówności można je zredukować do dwóch
            zmiennych:
            $$
                \begin{aligned}
                    3 \leq I_3 + I_4 \leq 5 \\
                    1 \leq 4 - I_3 \leq 3 \\
                    1 \leq I_3 \leq 3 \\
                    1 \leq I_4 \leq 3 
                \end{aligned}
            $$
            Można to już przedstawić na płaszczyźnie i policzyć wierzchołki wielokąta w ten sposób
            uzyskanego:
            \begin{figure}[H]
                \centering
                \def \svgwidth{0.7\columnwidth}
                \input{hexagon.pdf_tex}
                \caption{Wielokąt wyznaczany przez pozostałe zmienne}
            \end{figure}\noindent
            Potencjalnych rozwiązań jest więc sześć.
	\section{Wnioski z wykonanego ćwiczenia}
        Ćwiczenie te przyniosło sporo doświadczenia, jeżeli chodzi o wizualizację danych.
        Rozwiązanie poszczególnych zagadnień programowania liniowego jak i ich ułożenie było
        zdecydowanie prostsze od wizualizacji uzyskanych w ćwiczeniu wyników.
        \\ \\
        O ile ćwiczenie pierwsze nie wnosiło zbyt wiele do tematu, o tyle ćwiczenie drugie 
        pokazało jak można sprowadzić zadanie, które dla zadanych zmiennych optymalizacyjnych
        nie jest zadaniem programowania liniowego (rezystancje) do zadania programowania liniowego
        (wykorzystanie prądów), a w szczególności do jego postaci kanonicznej.
        \\ \\
        Cennym doświadczeniem było również zastosowanie dostarczonych wraz z bilbiotekami solwerów,
        gdyż pokazało to na czym w rzeczywistości polegają zagadnienia związane z optymalizacją.
\end{document}
