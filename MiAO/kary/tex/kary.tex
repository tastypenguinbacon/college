\documentclass[a4paper, 12pt]{article}
\usepackage[utf8]{inputenc}
\usepackage{geometry}
\usepackage{polski}
\usepackage{graphicx}
\usepackage{float}
\usepackage{etoolbox,refcount}
\usepackage{multicol}
\usepackage{fancyhdr}
\usepackage{listings}
\usepackage{amsmath}
\usepackage{tabularx}
\usepackage{svg}

\newgeometry{left=2.5cm, right=2.5cm, bottom=2.5cm, top=2.5cm}


\begin{document}
	\noindent
	\resizebox{\textwidth}{!}{
		\begin{tabular}{|c|c|c|c|}
			\hline 
			\multicolumn{4}{|l|}{Funkcja kary} \\ 
			\hline 
			Adrian Jałoszewski & 10 V 2017 & Środa 14:00 &  \\ 
			\hline 
		\end{tabular}
	} 
	\section{Cel ćwiczenia}
        Celem ćwiczenia jest zapoznanie się z metodami funkcji kary wewnętrznej oraz zewnętrznej oraz z
        ograniczeniami ich stosowalności.
	\section{Przebieg ćwiczenia}
        \subsection{Zadanie 1}
            Zadanie ma na celu minimalizować funkcję celu:
            $$
                f(x_1, x_2) = x_1^2 + x_2^2
            $$
            Przy ograniczeniu równościowym:
            $$
                x_2 = 1
            $$
            Podstawiając $x_2 = 1$ otrzymuje się równanie paraboli $x_1^2 + 1$, które posiada minimum
            dla $x_1 = 0$. Minimum wyznaczone analitycznie znajduje się więc w punkcie $(x_1, x_2) = 
            (0, 1)$.
            \\ \\
            Ponieważ mamy do czynienia z ograniczeniem równościowym, to należy je zapisać jako:
            $$
                \begin{aligned}
                    g_1(x) = x_2 - 1 \leq& 0 \\
                    g_2(x) = -x_2 + 1 \leq& leq 0
                \end{aligned}
            $$
            Ponieważ w funkcji kary występuje człon $\max(g_i(x), 0)$, a zawsze jedno z ograniczeń
            będzie niedodatnie można funkcję kary przedstawić jako:
            $$
                \Phi = 2^i (x_2 - 1) ^ 2
            $$
            \begin{figure}[H]
                \centering
                \def \svgwidth{0.7\columnwidth}
                \input{first_allowed.pdf_tex}
                \caption{Funkcja celu, obszar rozwiązań dopuszczalnych oraz punkty wyznaczone}
            \end{figure}\noindent
            Odległość wyznaczonych tą metodą punktów od rozwiązania analitycznego w zależności od
            iteracji jest przedstawiona na poniższym wykresie. Iteracje są tu przesunięte o 1, gdyż 
            zerowa iteracja przedstawia przypadek gdy nie ma funkcji kary.
            \begin{figure}[H]
                \centering
                \def \svgwidth{0.7\columnwidth}
                \input{first_distance.pdf_tex}
                \caption{Odległość od punktu startowego}
            \end{figure}\noindent
            \begin{figure}[H]
                \centering
                \def \svgwidth{0.7\columnwidth}
                \input{first_ratio.pdf_tex}
                \caption{Stosunek dwóch następujących po sobie wartości}
            \end{figure}\noindent
        \subsection{Zadanie 2}
            Zadanie jest źle postawione, ponieważ obydwie proste są do siebie równoległe i nie posiadają
            punktów wspólnych (są względem siebie przesunięte). 
            \\ \\
            Funkcja celu jest dana jako odległość od punktu (0, 0), jest to więc przeciwprostokątna
            trójkąta prostokątnego o bokach rownych współrzędnym $x_1$ oraz $x_2$. Jest to więc:
            $$
                f(x_1, x_2) = \sqrt{x_1^2 + x_2^2}
            $$
            Ograniczenia równościowe zadane w treści zadania: 
            $$
                \begin{aligned}
                    g_1(x) = x_1 + x_2 - 1 =& 0 \\
                    g_2(x) = x_1 + x_2 - 2 =& 0
                \end{aligned}
            $$
            Podobnie jak w poprzednim punkcie rozważania sprowadzają się do kwadratów każdego z
            ograniczeń.
            \begin{figure}[H]
                \centering
                \def \svgwidth{0.7\columnwidth}
                \input{second_scatter_points.pdf_tex}
                \caption{Wyznaczone punkty}
            \end{figure}\noindent
            Poszczególne punkty leżą na prostej przechodzącej przez środek i będącej prostopadłą
            do prostych stanowiących ograniczenia. 
        \subsection{Zadanie 3}
            Dana jest funkcja do minimalizacji:
            $$
                f(x_1, x_2) = \frac{1}{3} (x_1 + 1) ^3 + x_2
            $$
            Funkcja ta jest ograniczona przez:
            $$
                \begin{aligned}
                    g_1(x) =& x_1 - 1 \geq 0 \\
                    g_2(x) =& x_2 \geq 0 
                \end{aligned}
            $$
            Ponieważ każda funkcja składa się sumy dwóch rosnących funkcji (na zadanych ograniczeniach)
            jednej zmiennej, a na każdą ze zmiennych nałożone są niezależne niezależne od siebie 
            ograniczenia, to minimum znajduje się dla najmniejszych $x_1$ oraz $x_2$ w punkcie 
            $(x_1, x_2) = (1, 0)$. 
            \begin{figure}[H]
                \centering
                \def \svgwidth{0.7\columnwidth}
                \input{third_optimization.pdf_tex}
                \caption{Wyznaczanie minimum dla różnych parametrów}
            \end{figure}\noindent
            Można zauważyć, że im mniejszy współczynnik $c$, tym więcej iteracji jest potrzebnych do
            dokonania minimalizacji, gdyż współczynnik $k$ musi być więcej razy dzielony, aby uzyskać
            daną wartość.
            \begin{figure}[H]
                \centering
                \def \svgwidth{0.7\columnwidth}
                \input{third_iterations.pdf_tex}
                \caption{Liczba iteracji w zależności od wartości parametrów}
            \end{figure}\noindent
    \section{Wnioski i obserwacje z wykonanego ćwiczenia}
        \subsection{Zadanie 1}
            Optymalizacja przy pomocy zewnętrznej funkcji kary jest metodą bardzo szybko zbieżną ze
            względu na występujący czynnik wykładniczy. Satysfakcjonujący wynik jest uzyskiwany już
            po dziesięciu iteracjach. Użyta funkcja spełnia warunki bycia funkcją kary -- zeruje się 
            dla zbioru dopuszczalnego ($x_2 = 1$) oraz dla pozostałych punktów przyjmuje wartości
            dodatnie. Wraz ze wzrostem współczynnika k ($2^i$) rośnie jej wartość oraz nie jest
            ograniczona od góry.
        \subsection{Zadanie 2}
            Mimo tego, że zadanie jest źle postawione, funkcja kary pozwala na znalezienie wyniku,
            którego lokalizacja zależy od tego w jakim stopniu karamy rozwiązania bardziej odległe od
            prostych.
        \subsection{Zadanie 3}
            Metoda wewnętrznej funkcji kary jest zdecydowanie mniej skuteczna od metody zewnętrznej
            funkcji kary. Jest to spowodowane tym, że wiele metod optymalizujących funkcje jest 
            najzwyczajniej w stanie przeskoczyć narzucone przez nią ograniczenia, a po drugiej
            stronie ograniczeń funkcja ta potrafi nawet zmierzać do minus nieskończoności. Metoda ta
            zdaje się być skuteczna w przypadku, gdy mamy do czynienia z bardzo dużą rozdzielczością 
            poszukiwania, jednak i w tym przypadku zdecydowanie lepiej jest zastosować zewnętrzną
            funkcję kary. W celu wizualizacji danych byłem zmuszony tutaj zastosować pewnego rodzaju
            oszustwo, które polegało na podawaniu pewnej znacznie większej od pozostałych wartości,
            tak aby solvery nie były w stanie wejść na tamten obszar. Ponieważ zbiór $X_0$ jest tutaj
            niepusty, ograniczenia są ćwiartką płaszczyzny -- zbiór wypukły, a funkcja celu również 
            tworzy zbiór wypukły, a ich kombinacja tworzy zbiór ściśle wypukły, to jest to poprawna
            wewnętrzna funkcja celu. 
\end{document}
