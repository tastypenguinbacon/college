\documentclass[a4paper, 12pt]{article}
\usepackage[utf8]{inputenc}
\usepackage{geometry}
\usepackage{polski}
\usepackage{graphicx}
\usepackage{float}
\usepackage{etoolbox,refcount}
\usepackage{multicol}
\usepackage{fancyhdr}
\usepackage{listings}
\usepackage{amsmath}
\usepackage{tabularx}
\usepackage{svg}

\newgeometry{left=2.5cm, right=2.5cm, bottom=2.5cm, top=2.5cm}


\begin{document}
	\noindent
	\resizebox{\textwidth}{!}{
		\begin{tabular}{|c|c|c|c|}
			\hline 
			\multicolumn{4}{|l|}{Funkcja kary} \\ 
			\hline 
			Adrian Jałoszewski & 10 V 2017 & Środa 14:00 &  \\ 
			\hline 
		\end{tabular}
	} 
	\section{Cel ćwiczenia}
	\section{Przebieg ćwiczenia}
	\section{Wnioski z wykonanego ćwiczenia}
\end{document}
