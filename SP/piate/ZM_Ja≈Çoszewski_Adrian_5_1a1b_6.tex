\documentclass[a4paper, 12pt, titlepage]{article}
\usepackage[utf8]{inputenc}
\usepackage{geometry}
\usepackage{polski}
\usepackage{graphicx}
\usepackage{float}
\usepackage{etoolbox,refcount}
\usepackage{multicol}
\usepackage{fancyhdr}
\pagestyle{fancy}
\title{Projekt i realizacja tensometrycznych przetworników pomiarowych siły i massy z wykorzystaniem belki giętej i przemysłowego panelu wzmacniacza tensometrycznego MVD555}
\author{\textbf{Adrian Jałoszewski}, Tomasz Kotowski,\\Krzysztof Skolimowski, Monika Ścisło}
\date{17 października 2016, poniedziałek, $9^{\underline{30}}$}
\newgeometry{left=2.5cm, right=2.5cm, bottom=2.5cm, top=2.5cm}

\begin{document}
	\maketitle
	\tableofcontents
	\newpage
	\section{Cel ćwiczenia}
		Celem ćwiczenia było zapoznanie się z różnymi konfiguracjami mostków tensometrycznych, a następnie na podstawie wyliczeń ich parametrów pokazanie nam propagacji błędów wynikających z niedokładnych pomiarów oraz przedstawienie dokładniejszej metody skalowania torów pomiarowych. Kolejnym celem ćwiczenie było zapoznanie się z konfigurowaniem torów pomiarowych przy pomocy oprogramowania komputerowego.
	\section{Wykonanie ćwiczenia}
		\subsection{Pomiar masy lub siły metodą mostka niezrównoważonego}
			\subsubsection{konfiguracja pół-mostka z tensometrami wzdłużnymi}
			\subsubsection{konfiguracja pół-mostka z tensometrem wzdłużnym i poprzecznym}
			\subsubsection{konfiguracja pełnego mostka z tensometrami wzdłużnymi i poprzecznymi}
		\subsection{Skalowanie torów pomiarowych w oparciu o obliczone teoretyczne czułości}
		\subsection{Skalowanie torów pomiarowych w oparciu o wzorzec masy}
		\subsection{Konfigurowanie torów pomiarowych w oparciu o oprogramowanie Catman\textregistered\ Express}
		\subsection{Badanie wpływu rezystancji kabla na błędy pomiaru w torze pomiarowym}
	\section{Wnioski i podsumowanie}
		Ćwiczenie te pozwoliło nam na wyznaczenie czułości poszczególnych mostków tensometrycznych oraz zapoznanie się ze skalowaniem ich parametrów tak aby pozwoliły nam w połączeniu ze wzmacniaczem pomiarowym MVD555 na skonstruowanie wagi. Po skonfigurowaniu wzmacniacza okazało się jednak, że pomiary znacznie odstają od wagi wzorców masy. Jak się okazuje jest to wynikiem propagacji błędów pomiarowych z pomiarów jakie zostały nam podane w treści zadania.
		\\
		\\
		Zdecydowanie nieprzyjazny użytkownikowi interfejs użytkownika z jakim zetknęliśmy się w przypadku wzmacniacza pomiarowego -- strasznie dużo operacji potrzebnych dla ustawienia jednego parametru -- mogliśmy w kolejnym kroku ćwiczenia zastąpić oprogramowaniem Catman\textregistered Express. Pierwsza z metod wymagała od nas wielokrotnego powtarzania jednej czynności i w przypadku chwili nieuwagi należało wykonać całą operację od nowa. 
		\\
		\\
		Podpunkt z wpływem rezystancji kabla na błędy pomiaru w torze pomiarowym mogliśmy tylko rozważyć teoretycznie. Nie pozwolił nam na to czas jakim dysponowaliśmy na wykonanie laboratorium. %todo make some moar ;_____;
\end{document}