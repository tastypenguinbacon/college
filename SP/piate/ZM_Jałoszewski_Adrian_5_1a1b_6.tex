\documentclass[a4paper, 12pt, titlepage]{article}
\usepackage[utf8]{inputenc}
\usepackage{geometry}
\usepackage{polski}
\usepackage{graphicx}
\usepackage{float}
\usepackage{etoolbox,refcount}
\usepackage{multicol}
\usepackage{fancyhdr}
\pagestyle{fancy}
\title{Projekt i realizacja tensometrycznych przetworników pomiarowych siły i massy z wykorzystaniem belki giętej i przemysłowego panelu wzmacniacza tensometrycznego MVD555}
\author{\textbf{Adrian Jałoszewski}, Tomasz Kotowski,\\Krzysztof Skolimowski, Monika Ścisło}
\date{17 października 2016, poniedziałek, $9^{\underline{30}}$}
\newgeometry{left=2.5cm, right=2.5cm, bottom=2.5cm, top=2.5cm}

\begin{document}
	\maketitle
	\tableofcontents
	\newpage
	\section{Cel ćwiczenia}
		Celem ćwiczenia było pokazanie nam propagacji błędów wynikających z niedokładnych pomiarów.
	\section{Wykaz aparatury}
		
	\section{Wykonanie ćwiczenia}
		\subsection{Pomiar masy lub siły metodą mostka niezrównoważonego}
			\subsubsection{konfiguracja pół-mostka z tensometrami wzdłużnymi}
			\subsubsection{konfiguracja pół-mostka z tensometrem wzdłużnym i poprzecznym}
			\subsubsection{konfiguracja pełnego mostka z tensometrami wzdłużnymi i poprzecznymi}
		\subsection{Skalowanie torów pomiarowych w oparciu o obliczone teoretyczne czułości}
		\subsection{Skalowanie torów pomiarowych w oparciu o wzorzec masy}
		\subsection{Konfigurowanie torów pomiarowych w oparciu o oprogramowanie Catman\textregistered\ Express}
	\section{Wnioski i podsumowanie}
		
\end{document}