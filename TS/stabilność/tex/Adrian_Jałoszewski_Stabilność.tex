\documentclass[a4paper, 12pt]{article}
\usepackage[utf8]{inputenc}
\usepackage{geometry}
\usepackage{polski}
\usepackage{graphicx}
\usepackage{float}
\usepackage{etoolbox,refcount}
\usepackage{multicol}
\usepackage{fancyhdr}
\usepackage{listings}
\usepackage{amsmath}
\usepackage{svg}

\title{Częstotliwościowe kryteria stabilności}
\author{Adrian Jałoszewski}
\date{21 marca 2017, godzina 12:30}
\newgeometry{left=2.5cm, right=2.5cm, bottom=2.5cm, top=2.5cm}


\begin{document}
	\lstset{language=Python, basicstyle=\footnotesize,
		keepspaces=true,frame=single,tabsize=4}
	\maketitle
	\section{Wstęp}
		Celem laboratorium było zapoznanie się z zastosowaniem częstotliwościowych kryteriów stabilności. Przykłady obejmowały kryterium Nyquista w zastosowaniu dla układu, który jest niestabilny bez sprzężenia zwrotnego, przypadku obiektu inercyjnego pierwszego rzędu z opóźnieniem oraz zadania dodatkowego, gdzie należy wyznaczyć dla jakiego wzmocnienia układ znajduje się na granicy stabilności.
	\section{Wykonanie}
		\subsection{Zadanie 2.1}
		Transmitancja jest zadana jako:
		$$
			G_0(s) = \frac{s + 1}{0,01s^4 + 0,5s^3 + 3 s^2 -10s + 10}
		$$ \noindent
		Należy wyznaczyć krytyczne wartości współczynnika wzmocnienia $K$ dla jakiego $K \cdot G_0(s)$ po zapięciu sprzężenia zwrotnego jest stabilne.
		\\
		\\
		Najpierw sprawdzam pierwiastki wielomianu charakterystycznego układu aby sprawdzić ile z nich posiada dodatnią liczb rzeczywistą. Pierwiastki wielomianu: 
		$$
			M(s) = 0,01s^4 + 0,5s^3 + 3 s^2 -10s + 10
		$$
		są w przybliżeniu następujące $s_0 \in \{-42,3443, -10,2412, 1,29278 \pm j\cdot 0,79665\}$. Dwa \linebreak pierwiastki mają więc dodatnią część rzeczywistą, dlatego też przyrost \linebreak $\Delta_{0 \leq \omega \leq \infty} \arg(1 + G_0(j\omega)) = \pi$. \\ \\
		\begin{figure}[H]
			\centering
			\def \svgwidth{0.7\columnwidth}
			\input{zad_1_first.pdf_tex}
			\caption{Wykres Nyquista dla $G_0(j\omega)$}
		\end{figure}\noindent
		Wykres ten oś rzeczywistą dla wartości rzeczywistych $-0,07074$ oraz $-0,007642$. Dlatego współczynnik $K$ musi się znajdować w przedziale około:
		$$
			K \in [14.14, 130.86]
		$$
		Dowodem tego są oscylacje nietłumione dla dokładniej wyznaczonych wartości:
		\begin{figure}[H]
			\centering
			\def \svgwidth{0.49\columnwidth}
			\input{zad_1_snd.pdf_tex}
			\def \svgwidth{0.49\columnwidth}
			\input{zad_1_thrd.pdf_tex}
			\caption{Odpowiedzi skokowe dla układu ze sprzężeniem zwrotnym}
		\end{figure}\noindent
		\newpage
		\subsection{Zadanie 2.2}
			Transmitancja dana wzorem:
			$$
				G(s) = \frac{4 e^ {-0.5s}}{s + 1}
			$$
			Podstawiając $s = j\omega$. I zapisując w postaci trygonometrycznej otrzymuję.
			$$
				G(j\omega) = \frac{4}{1 + \omega^2}
				\cdot(\cos 0.5\omega - \omega \cos 0.5\omega) - j(\sin0.5\omega + )
			$$
			\begin{figure}[H]
				\centering
				\def \svgwidth{0.7\columnwidth}
				\input{drugie.pdf_tex}
				\caption{Wykres Nyquista dla $G(j\omega)$}
			\end{figure}\noindent
			Układ jest niestabilny.
		\subsection{Zadanie dodatkowe}
		Zadanie dodatkowe polegało na wyznaczeniu wzmocnienia krytycznego dla układu o transmitancji:
		$$
			G_0(s) = \frac{1}{s(T_1s+1)(T_2s+1)}
		$$
		Gdzie w szczególnym przypadku $T_1 = 1$, $T_2 = 2$. Więc wszystkie pierwiastki wielomianu charakterystycznego są niedodatnie, więc wykres nie może obejmować punktu $(-1, j0)$
		\begin{figure}[H]
			\centering
			\def \svgwidth{0.7\columnwidth}
			\input{dodatkowe.pdf_tex}
			\caption{Wykres Nyquista dla $G_0(j\omega)$}
		\end{figure}\noindent
		Punkt przecięcia wykresu z osią rzeczywistą wypada w $-\frac{2}{3}$, dlatego też wzmocnienie krytyczne wynosi $K = 1,5$.
		\\ \\
		Do wyznaczenia miejsc zerowych ujemnych posłużyła następująca funkcja.
\begin{lstlisting}
def extract_from_poly(num, den):
	numerator = np.poly1d(num)
	num_re, num_im = real_imag_poly(numerator)
	num_re, num_im = np.poly1d(num_re), np.poly1d(num_im)
	denominator = np.poly1d(den)
	den_re, den_im = real_imag_poly(denominator)
	den_re, den_im = np.poly1d(den_re), np.poly1d(den_im)
	multiplier = den_re - den_im * 1j
	new_num = num_re + num_im * 1j
	new_num *= multiplier
	new_num_im = np.poly1d(np.imag(new_num.coeffs))
	roots = np.roots(new_num_im)
	roots = [root for root in roots if root.real >= 0]
	print(roots)
	real_parts = np.polyval(new_num, roots) / \
	    (np.polyval(den_re, roots) *
	    np.polyval(den_re, roots) +
	    np.polyval(den_im, roots) *
	    np.polyval(den_im, roots))
	return np.array([x for x in real_parts.real if x < 0])
\end{lstlisting}
		Posiłkowała się ona funkcją, która dzieli wielomian na jego część rzeczywistą i urojoną:
\begin{lstlisting}
def real_imag_poly(polynomial: np.poly1d):
	pol = polynomial.coeffs
	rev = [x for x in reversed(pol)]
	sign = 1
	real_ans = []
	imag_ans = []
	for i in range(0, len(rev)):
		if i % 2 == 0:
			real_ans.append(rev[i] * sign)
			imag_ans.append(0)
		else:
			real_ans.append(0)
			imag_ans.append(rev[i] * sign)
			sign *= -1
	real_ans = [x for x in reversed(real_ans)]
	imag_ans = [x for x in reversed(imag_ans)]
	return real_ans, imag_ans
\end{lstlisting}
	\section{Wnioski}
		
\end{document}
