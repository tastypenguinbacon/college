\documentclass[a4paper, 10pt]{article}
\usepackage[utf8]{inputenc}
\usepackage{geometry}
\usepackage{polski}
\usepackage{graphicx}
\usepackage{float}
\usepackage{etoolbox,refcount}
\usepackage{multicol}
\usepackage{fancyhdr}
\usepackage{listings}
\usepackage{amsmath}
\usepackage{svg}

\title{Pierwsza metoda Lapunowa}
\author{Adrian Jałoszewski}
\date{4 kwietnia 2017, godzina 12:30}
\newgeometry{left=2.5cm, right=2.5cm, bottom=2cm, top=2cm}


\begin{document}
	\maketitle
	\section{Wstęp}
		Celem laboratorium było zapoznanie się z układami nieliniowymi oraz ich przybliżeniami liniowymi. //TODO
	\section{Wykonanie}
		\subsection{Układ równań van der Polla}
			\subsubsection{Współczynnik a = 1}
				\begin{figure}[H]
					\centering
					\def \svgwidth{0.75\columnwidth}
					\input{van_der_poll_1.pdf_tex}
					\caption{Układ nieliniowy}
				\end{figure}\noindent


				\begin{figure}[H]
					\centering
					\def \svgwidth{0.75\columnwidth}
					\input{van_der_poll_1_lin.pdf_tex}
					\caption{Układ zlinearyzowany}
				\end{figure}\noindent

				\begin{figure}[H]
					\centering
					\def \svgwidth{0.75\columnwidth}
					\input{van_der_poll_1_both_close.pdf_tex}
					\caption{Porównanie w otoczeniu punktu równowagi}
				\end{figure}\noindent

			\subsubsection{Współczynnik a = 5}
				\begin{figure}[H]
					\centering
					\def \svgwidth{0.75\columnwidth}
					\input{van_der_poll_5.pdf_tex}
					\caption{Układ nieliniowy}
				\end{figure}\noindent


				\begin{figure}[H]
					\centering
					\def \svgwidth{0.75\columnwidth}
					\input{van_der_poll_5_lin.pdf_tex}
					\caption{Układ zlinearyzowany}
				\end{figure}\noindent

				\begin{figure}[H]
					\centering
					\def \svgwidth{0.75\columnwidth}
					\input{van_der_poll_5_both_close.pdf_tex}
					\caption{Porównanie w otoczeniu punktu równowagi}
				\end{figure}\noindent

			\subsubsection{Współczynnik a = 0}
				\begin{figure}[H]
					\centering
					\def \svgwidth{0.75\columnwidth}
					\input{van_der_poll_0.pdf_tex}
					\caption{Układ nieliniowy}
				\end{figure}\noindent


			\subsubsection{Współczynnik a = 0,5}
				\begin{figure}[H]
					\centering
					\def \svgwidth{0.75\columnwidth}
					\input{van_der_poll_0_5.pdf_tex}
					\caption{Układ nieliniowy}
				\end{figure}\noindent


				\begin{figure}[H]
					\centering
					\def \svgwidth{0.75\columnwidth}
					\input{van_der_poll_0_5_lin.pdf_tex}
					\caption{Układ zlinearyzowany}
				\end{figure}\noindent


				\begin{figure}[H]
					\centering
					\def \svgwidth{0.75\columnwidth}
					\input{van_der_poll_0_5_both_close.pdf_tex}
					\caption{Porównanie w otoczeniu punktu równowagi}
				\end{figure}\noindent

		\subsection{Wahadło tłumione}
			\subsubsection{Współczynniki a=1, b=1}
				\begin{figure}[H]
					\centering
					\def \svgwidth{0.75\columnwidth}
					\input{pendulum_1.pdf_tex}
					\caption{Układ nieliniowy}
				\end{figure}\noindent


				\begin{figure}[H]
					\centering
					\def \svgwidth{0.75\columnwidth}
					\input{pendulum_1_lin.pdf_tex}
					\caption{Układ zlinearyzowany}
				\end{figure}\noindent


				\begin{figure}[H]
					\centering
					\def \svgwidth{0.75\columnwidth}
					\input{pendulum_1_both_close.pdf_tex}
					\caption{Porównanie w otoczeniu punktu równowagi}
				\end{figure}\noindent

			\subsubsection{Współczynnik a = 4, b = 1}
				\begin{figure}[H]
					\centering
					\def \svgwidth{0.75\columnwidth}
					\input{pendulum_2.pdf_tex}
					\caption{Układ nieliniowy}
				\end{figure}\noindent


				\begin{figure}[H]
					\centering
					\def \svgwidth{0.75\columnwidth}
					\input{pendulum_2_lin.pdf_tex}
					\caption{Układ zlinearyzowany}
				\end{figure}\noindent



				\begin{figure}[H]
					\centering
					\def \svgwidth{0.75\columnwidth}
					\input{pendulum_2_both_close.pdf_tex}
					\caption{Porównanie w otoczeniu punktu równowagi}
				\end{figure}\noindent

			\subsubsection{Współczynnik a = 4, b = 0}
				\begin{figure}[H]
					\centering
					\def \svgwidth{0.75\columnwidth}
					\input{pendulum_3.pdf_tex}
					\caption{Układ nieliniowy}
				\end{figure}\noindent

			\subsection{Współczynnik a = 1, b = 3}
				\begin{figure}[H]
					\centering
					\def \svgwidth{0.75\columnwidth}z
					\input{pendulum_4.pdf_tex}
					\caption{Układ nieliniowy}
				\end{figure}\noindent


				\begin{figure}[H]
					\centering
					\def \svgwidth{0.75\columnwidth}
					\input{pendulum_4_lin.pdf_tex}
					\caption{Układ zlinearyzowany}
				\end{figure}\noindent


				\begin{figure}[H]
					\centering
					\def \svgwidth{0.75\columnwidth}
					\input{pendulum_4_both_close.pdf_tex}
					\caption{Porównanie w otoczeniu punktu równowagi}
				\end{figure}\noindent

		\subsection{Układ mechaniczny}
			\subsubsection{Sprężyna twarda}
				\begin{figure}[H]
					\centering
					\def \svgwidth{0.75\columnwidth}
					\input{mechanical_1.pdf_tex}
					\caption{Układ nieliniowy}
				\end{figure}\noindent


				\begin{figure}[H]
					\centering
					\def \svgwidth{0.75\columnwidth}
					\input{mechanical_1_lin.pdf_tex}
					\caption{Układ zlinearyzowany}
				\end{figure}\noindent


				\begin{figure}[H]
					\centering
					\def \svgwidth{0.75\columnwidth}
					\input{mechanical_1_both_close.pdf_tex}
					\caption{Porównanie w otoczeniu punktu równowagi}
				\end{figure}\noindent

			\subsubsection{Sprężyna miękka}
				\begin{figure}[H]
					\centering
					\def \svgwidth{0.75\columnwidth}
					\input{mechanical_2.pdf_tex}
					\caption{Układ nieliniowy}
				\end{figure}\noindent


				\begin{figure}[H]
					\centering
					\def \svgwidth{0.75\columnwidth}
					\input{mechanical_2_lin.pdf_tex}
					\caption{Układ zlinearyzowany}
				\end{figure}\noindent


				\begin{figure}[H]
					\centering
					\def \svgwidth{0.75\columnwidth}
					\caption{Porównanie w otoczeniu punktu równowagi}
				\end{figure}\noindent


				\begin{figure}[H]
					\centering
					\def \svgwidth{0.75\columnwidth}
					\input{mechanical_2_both_close.pdf_tex}
					\caption{Porównanie w otoczeniu punktu równowagi}
				\end{figure}\noindent


	\section{Wnioski}
		Mimo tego, że wiele układów ma te same jakobiany oraz te same punkty równowagi, to jednak mogą zachowywać się skrajnie różnie. Najlepszym przykładem tego jest układ mechaniczny ze sprężyną miękką oraz sprężyną twardą,
		gdzie dobór współczynnika, który jest przy członie zerującym się przy procesie linearyzacji ma wpływ na to czy układ zachowuje się asymptotycznie stabilnie, czy dla niektórych punktów startowych jego trajektoria jest
		nieograniczona.
		\\ \\
		Im rozważane otoczenie punków równowagi jest mniejsze, tym lepiej da się je przybliżyć modelami liniowymi. Wszystkie rozważane przypadki, które posiadały punkt równowagi, były w jego otoczeniu dobrze aproksymowane
		modelem liniowym.
		\\ \\
		Niektóre systemy nieliniowe posiadają wiele punktów równowagi -- co nie ma miejsca w układach liniowych. Dobrym przykładem tego jest wahadło, które posiada asymptotycznie stabilne punkty równowagi rozmieszczone w punktach
		$(0, k \cdot 2\pi)$ gdzie $k \in \mathbb{Z}$ oraz niestabilne w $(0, (2k + 1)\pi)$ -- na granicy obszarów przyciągania. Z modelu liniowego wynika, że układ posiada tylko jeden punkt równowagi i do niego będzie asymptotycznie
		zmierzał. W przypadku wahadła jest to efekt niezauważalny ludzkim okiem, ale przyrząd pomiarowy liczący bezwzględny kąt obrotu będzie podawał na zaprojektowany pod liniowy model regulator błędne informacje.
\end{document}
