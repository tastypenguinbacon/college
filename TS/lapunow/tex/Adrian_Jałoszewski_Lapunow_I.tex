\documentclass[a4paper, 10pt]{article}
\usepackage[utf8]{inputenc}
\usepackage{geometry}
\usepackage{polski}
\usepackage{graphicx}
\usepackage{float}
\usepackage{etoolbox,refcount}
\usepackage{multicol}
\usepackage{fancyhdr}
\usepackage{amsfonts}
\usepackage{listings}
\usepackage{amsmath}
\usepackage{svg}

\title{Pierwsza metoda Lapunowa}
\author{Adrian Jałoszewski}
\date{4 kwietnia 2017, godzina 12:30}
\newgeometry{left=2.5cm, right=2.5cm, bottom=2cm, top=2cm}


\begin{document}
	\maketitle
	\section{Wstęp}
		Celem laboratorium było zapoznanie się z układami nieliniowymi oraz ich przybliżeniami liniowymi. //TODO
	\section{Wykonanie laboratorium}
		\subsection{Układ równań van der Polla}
			Układ równań Van der Pola jest dany jako:
			$$
				\begin{aligned}
					\dot{x}_1(t) & = x_2(t) - x_1^3(t) - a x_1(t) \\
					\dot{x}_2(t) & = - x_1(t)
				\end{aligned}
			$$
			Punkty równowagi wyznaczam podstawiając za pochodne 0:
				$$
				\begin{aligned}
					0 & = x_2 - x_1^3 - a x_1 \\
					0 & = -x_1
				\end{aligned}
				$$
			Wynika z tego, że $x_1 = x_2 = 0$. Jakobian tego systemu jest następujący:
			$$
				J = \begin{bmatrix}
					-3x_1^2 -a & 1 \\
					-1 & 0
				\end{bmatrix}
			$$
			W punkcie równowagi macierz A ma postać:
			$$
				A = \begin{bmatrix}
				-a & 1 \\
				-1 & 0
				\end{bmatrix}
			$$
			Wielomian charakterystyczny macierzy A:
			$$
				\det(\lambda I - A) = \lambda ^ 2 + a \lambda + 1
			$$
			Dla poszczególnych wartości współczynnika $a$ własnych:
			\begin{itemize}
				\item[] $a = 0$ -- środek 
				\item[] $a \in (0, 2)$ -- ognisko asymptotycznie stabilne
				\item[] $a = 2$ -- węzeł zdegenerowany asymptotycznie stabilny
				\item[]	$a > 2$ -- węzeł asymptotycznie stabilny
			\end{itemize}
			
			\subsubsection{Współczynnik a = 1}
				Jest to przypadek kiedy zlinearyzowany układ posiada portret fazowy typu ogniska asymptotycznie stabilnego o macierzy stanu:
				$$
				A = \begin{bmatrix}
					-1 & 1 \\
					-1 & 0
				\end{bmatrix}
				$$
				Wartości własne macierzy: $\lambda_1 = -\frac{1}{2} + \frac{\sqrt{3}}{2}$, $\lambda_2 = -\frac{1}{2} - \frac{\sqrt{3}}{2}$, układ jest więc asymptotycznie stabilny w okolicy punktu (0, 0)
				\begin{figure}[H]
					\centering
					\def \svgwidth{0.75\columnwidth}
					\input{van_der_poll_1.pdf_tex}
					\caption{Układ nieliniowy}
				\end{figure}\noindent


				\begin{figure}[H]
					\centering
					\def \svgwidth{0.75\columnwidth}
					\input{van_der_poll_1_lin.pdf_tex}
					\caption{Układ zlinearyzowany}
				\end{figure}\noindent

				\begin{figure}[H]
					\centering
					\def \svgwidth{0.75\columnwidth}
					\input{van_der_poll_1_both_close.pdf_tex}
					\caption{Porównanie w otoczeniu punktu równowagi}
				\end{figure}\noindent
			\subsubsection{Współczynnik a = 2}
					Jest to przypadek kiedy zlinearyzowany układ posiada portret fazowy typu węzła zdegenerowanego asymptotycznie stabilnego o macierzy stanu:
					$$
					A = \begin{bmatrix}
					-2 & 1 \\
					-1 & 0
					\end{bmatrix}
					$$
					Wartości własne macierzy: $\lambda_{1/2} = -1$, układ jest więc asymptotycznie stabilny w okolicy punktu (0, 0)
				\begin{figure}[H]
					\centering
					\def \svgwidth{0.7\columnwidth}
					\input{van_der_poll_2.pdf_tex}
					\caption{Układ nieliniowy}
				\end{figure}\noindent
				
				\begin{figure}[H]
					\centering
					\def \svgwidth{0.7\columnwidth}
					\input{van_der_poll_2_lin.pdf_tex}
					\caption{Układ zlinearyzowany}
				\end{figure}\noindent
				
				
				\begin{figure}[H]
					\centering
					\def \svgwidth{0.7\columnwidth}
					\input{van_der_poll_2_both_close.pdf_tex}
					\caption{Porównanie w otoczeniu punktu równowagi}
				\end{figure}\noindent
			\subsubsection{Współczynnik a = $\frac{5}{2}$}
				Współczynnik został dobrany tak aby trajektorie układu dla macierzy w postaci kanonicznej były postaci $x_2 = x_1^4$ -- stosunek wartości własnych powinien być równy 4. Stopień wielomianu został wybrany w ten sposób aby krzywizna trajektorii była dobrze widoczna na portretach fazowych -- (dla wielomianu drugiego stopnia $a=\frac{3}{\sqrt{2}} \approx 2,12$, dla trzeciego stopnia $a=\frac{4}{\sqrt{3}} \approx 2,31$ -- wartości na tyle bliskie tym z przypadku węzła zdegenerowanego, że trudno było rozróżnić poszczególne przypadki).
				$$
					A = \begin{bmatrix}
						-5/2 & 1 \\
						-1 & 0
					\end{bmatrix}
				$$
				Wartości własne macierzy: $\lambda_1 = -\frac{1}{2}$, $\lambda_2 = -2$, układ jest więc asymptotycznie stabilny w okolicy punktu (0, 0)
				\begin{figure}[H]
					\centering
					\def \svgwidth{0.75\columnwidth}
					\input{van_der_poll_5.pdf_tex}
					\caption{Układ nieliniowy}
				\end{figure}\noindent
				\begin{figure}[H]
					\centering
					\def \svgwidth{0.75\columnwidth}
					\input{van_der_poll_5_lin.pdf_tex}
					\caption{Układ zlinearyzowany}
				\end{figure}\noindent

				\begin{figure}[H]
					\centering
					\def \svgwidth{0.75\columnwidth}
					\input{van_der_poll_5_both_close.pdf_tex}
					\caption{Porównanie w otoczeniu punktu równowagi}
				\end{figure}\noindent

			\subsubsection{Współczynnik a = 0}
				Jest to przypadek kiedy zlinearyzowany układ posiada portret fazowy typu koła o macierzy stanu:
				$$
				A = \begin{bmatrix}
				0 & 1 \\
				-1 & 0
				\end{bmatrix}
				$$
				Wartości własne $\lambda_1 = -j$, $\lambda_2 = j$ mają zerową część rzeczywistą -- o stabilności układu nieliniowego nie można wnioskować. Jest to przypadek niespełniający warunków opisanych w poleceniu, więc nie będzie szczególnie rozważany.
				\begin{figure}[H]
					\centering
					\def \svgwidth{0.75\columnwidth}
					\input{van_der_poll_0.pdf_tex}
					\caption{Układ nieliniowy}
				\end{figure}\noindent
		\subsection{Wahadło tłumione}
			Wahadło tłumione jest zadane wzorem:
			$$
				\ddot{y}(t) + \frac{g}{l} \sin y(t) + \frac{c}{lm} \dot{y}(t) = 0
			$$
			Dla uproszczenia rozważań podstawiam $a = \frac{g}{l}$, $b = \frac{c}{lm}$, otrzymuję w ten sposób:
			$$
				\ddot{y}(t) + a \sin y(t) + b \dot{y}(t) = 0
			$$
			Ze względu na charakter parametrów \textit{g, l, c, m} parametry \textit{a, b} są większe lub równe 0. Układ ten można przedstawić równoważnie:
			$$
				\begin{aligned}
					\dot{x}_1 & = x_2 \\
					\dot{x}_2 & = - b x_2 - a \sin x_1 
				\end{aligned}
			$$
			Gdzie $x_2 = \dot{y}$ oraz $x_1 = y$. Aby wyznaczyć punkty równowagi podstawiam za pochodne 0.
			$$
				\begin{aligned}
					0 & = x_2 \\
					0 & = - b x_2 - a \sin x_1
				\end{aligned}
			$$
			Punkty równowagi następują więc dla $(x_1, x_2) = (k \pi, 0)$, gdzie $k \in \mathbb{Z}$. Jakobian dla tego układu jest następujący:
			$$
				J = \begin{bmatrix}
					0 & 1 \\
					- a \cos x_1 & -b
				\end{bmatrix}
			$$
			Należy rozważyć dwa przypadki punktów stabilności:
			\begin{itemize}
				\item[] $x_2 = 0$, $x_1 = 2k \pi$ dla $k \in \mathbb{Z}$ -- wtedy $\cos x_1 = 1$
					$$
						A = \begin{bmatrix}
							0 & 1 \\
							- a  & -b
						\end{bmatrix}
					$$
					Wielomian charakterystyczny macierzy A jest dany jako: $\lambda^2 + b\lambda + a$, jeżeli wartości \textit{a, b} są dodatnie, to wielomian drugiego rzędu posiada pierwiastki w lewej półpłaszczyźnie zespolonej -- są to punkty równowagi asymptotycznie stabilne.
					Wielomian charakterystyczny macierzy A dany jest jako $\lambda^2 + \lambda b$
				\item[] $x_2 = 0$, $x_1 = (2k + 1) \pi$ dla $k \in \mathbb{Z}$ -- wtedy $\cos x_1 = -1$
					$$
						A = \begin{bmatrix}
							0 & 1 \\
							a  & -b
						\end{bmatrix}
					$$
					Wielomian charakterystyczny macierzy A jest dany jako: $\lambda^2 + b\lambda - a$, gdy parametry \textit{a, b}, ponieważ badamy układ nieliniowy, to współczynnik $a$ musi być większy od zera -- wyraz wolny wielomianu jest zawsze mniejszy od zera, z czego wynika, że w tych punktach występuje równowaga niestabilna.
			\end{itemize}
			Ponieważ układ zachowuje się cyklicznie względem $x_1$ z okresem $2 \pi$, wystarczy w rozważaniach układu zlinearyzowanego wziąć pod uwagę tylko punkt (0, 0).
			\subsubsection{Współczynniki a=1, b=1}
				\begin{figure}[H]
					\centering
					\def \svgwidth{0.75\columnwidth}
					\input{pendulum_1.pdf_tex}
					\caption{Układ nieliniowy}
				\end{figure}\noindent


				\begin{figure}[H]
					\centering
					\def \svgwidth{0.75\columnwidth}
					\input{pendulum_1_lin.pdf_tex}
					\caption{Układ zlinearyzowany}
				\end{figure}\noindent


				\begin{figure}[H]
					\centering
					\def \svgwidth{0.75\columnwidth}
					\input{pendulum_1_both_close.pdf_tex}
					\caption{Porównanie w otoczeniu punktu równowagi}
				\end{figure}\noindent

			\subsubsection{Współczynnik a = 4, b = 1}
				\begin{figure}[H]
					\centering
					\def \svgwidth{0.75\columnwidth}
					\input{pendulum_2.pdf_tex}
					\caption{Układ nieliniowy}
				\end{figure}\noindent


				\begin{figure}[H]
					\centering
					\def \svgwidth{0.75\columnwidth}
					\input{pendulum_2_lin.pdf_tex}
					\caption{Układ zlinearyzowany}
				\end{figure}\noindent



				\begin{figure}[H]
					\centering
					\def \svgwidth{0.75\columnwidth}
					\input{pendulum_2_both_close.pdf_tex}
					\caption{Porównanie w otoczeniu punktu równowagi}
				\end{figure}\noindent

			\subsubsection{Współczynnik a = 4, b = 0}
				\begin{figure}[H]
					\centering
					\def \svgwidth{0.75\columnwidth}
					\input{pendulum_3.pdf_tex}
					\caption{Układ nieliniowy}
				\end{figure}\noindent

			\subsection{Współczynnik a = 1, b = 3}
				\begin{figure}[H]
					\centering
					\def \svgwidth{0.75\columnwidth}z
					\input{pendulum_4.pdf_tex}
					\caption{Układ nieliniowy}
				\end{figure}\noindent


				\begin{figure}[H]
					\centering
					\def \svgwidth{0.75\columnwidth}
					\input{pendulum_4_lin.pdf_tex}
					\caption{Układ zlinearyzowany}
				\end{figure}\noindent


				\begin{figure}[H]
					\centering
					\def \svgwidth{0.75\columnwidth}
					\input{pendulum_4_both_close.pdf_tex}
					\caption{Porównanie w otoczeniu punktu równowagi}
				\end{figure}\noindent

		\subsection{Układ mechaniczny}
			\subsubsection{Sprężyna twarda}
				\begin{figure}[H]
					\centering
					\def \svgwidth{0.75\columnwidth}
					\input{mechanical_1.pdf_tex}
					\caption{Układ nieliniowy}
				\end{figure}\noindent


				\begin{figure}[H]
					\centering
					\def \svgwidth{0.75\columnwidth}
					\input{mechanical_1_lin.pdf_tex}
					\caption{Układ zlinearyzowany}
				\end{figure}\noindent


				\begin{figure}[H]
					\centering
					\def \svgwidth{0.75\columnwidth}
					\input{mechanical_1_both_close.pdf_tex}
					\caption{Porównanie w otoczeniu punktu równowagi}
				\end{figure}\noindent

			\subsubsection{Sprężyna miękka}
				\begin{figure}[H]
					\centering
					\def \svgwidth{0.75\columnwidth}
					\input{mechanical_2.pdf_tex}
					\caption{Układ nieliniowy}
				\end{figure}\noindent


				\begin{figure}[H]
					\centering
					\def \svgwidth{0.75\columnwidth}
					\input{mechanical_2_lin.pdf_tex}
					\caption{Układ zlinearyzowany}
				\end{figure}\noindent


				\begin{figure}[H]
					\centering
					\def \svgwidth{0.75\columnwidth}
					\caption{Porównanie w otoczeniu punktu równowagi}
				\end{figure}\noindent


				\begin{figure}[H]
					\centering
					\def \svgwidth{0.75\columnwidth}
					\input{mechanical_2_both_close.pdf_tex}
					\caption{Porównanie w otoczeniu punktu równowagi}
				\end{figure}\noindent


	\section{Wnioski}
		Mimo tego, że wiele układów ma te same jakobiany oraz te same punkty równowagi, to jednak mogą zachowywać się skrajnie różnie. Najlepszym przykładem tego jest układ mechaniczny ze sprężyną miękką oraz sprężyną twardą,
		gdzie dobór współczynnika, który jest przy członie zerującym się przy procesie linearyzacji ma wpływ na to czy układ zachowuje się asymptotycznie stabilnie, czy dla niektórych punktów startowych jego trajektoria jest
		nieograniczona.
		\\ \\
		Im rozważane otoczenie punków równowagi jest mniejsze, tym lepiej da się je przybliżyć modelami liniowymi. Wszystkie rozważane przypadki, które posiadały punkt równowagi, były w jego otoczeniu dobrze aproksymowane
		modelem liniowym.
		\\ \\
		Niektóre systemy nieliniowe posiadają wiele punktów równowagi -- co nie ma miejsca w układach liniowych. Dobrym przykładem tego jest wahadło, które posiada asymptotycznie stabilne punkty równowagi rozmieszczone w punktach
		$(0, k \cdot 2\pi)$ gdzie $k \in \mathbb{Z}$ oraz niestabilne w $(0, (2k + 1)\pi)$ -- na granicy obszarów przyciągania. Z modelu liniowego wynika, że układ posiada tylko jeden punkt równowagi i do niego będzie asymptotycznie
		zmierzał. W przypadku wahadła jest to efekt niezauważalny ludzkim okiem, ale przyrząd pomiarowy liczący bezwzględny kąt obrotu będzie podawał na zaprojektowany pod liniowy model regulator błędne informacje.
\end{document}
