\documentclass[a4paper, 10pt]{article}
\usepackage[utf8]{inputenc}
\usepackage{geometry}
\usepackage{polski}
\usepackage{graphicx}
\usepackage{float}
\usepackage{etoolbox,refcount}
\usepackage{multicol}
\usepackage{fancyhdr}
\usepackage{listings}
\usepackage{amsmath}
\usepackage{svg}

\title{Pierwsza metoda Lapunowa}
\author{Adrian Jałoszewski}
\date{4 kwietnia 2017, godzina 12:30}
\newgeometry{left=2.5cm, right=2.5cm, bottom=2cm, top=2cm}


\begin{document}
	\maketitle
	\tableofcontents
	\section{Wstęp}
		Celem laboratorium było zapoznanie się z regulatorami P, PI, PD oraz PID w dziedzinie częstotliwości, traktując je jako filtry. Oprócz tego należało zaprojektować filtr pasmowoprzepustowy.
	\section{Wykonanie}
		\subsection{Układ równań van der Polla}
			\subsubsection{Współczynnik a = 1}
				\begin{figure}[H]
					\centering
					\def \svgwidth{0.75\columnwidth}
					\input{van_der_poll_1.pdf_tex}
					\caption{Układ nieliniowy}
				\end{figure}\noindent
				
				
				\begin{figure}[H]
					\centering
					\def \svgwidth{0.75\columnwidth}
					\input{van_der_poll_1_lin.pdf_tex}
					\caption{Układ zlinearyzowany}
				\end{figure}\noindent
				
				\begin{figure}[H]
					\centering
					\def \svgwidth{0.75\columnwidth}
					\input{van_der_poll_1_both_close.pdf_tex}
					\caption{TODO}
				\end{figure}\noindent
			
			\subsubsection{Współczynnik a = 5}
				\begin{figure}[H]
					\centering
					\def \svgwidth{0.75\columnwidth}
					\input{van_der_poll_5.pdf_tex}
					\caption{Układ nieliniowy}
				\end{figure}\noindent
				
				
				\begin{figure}[H]
					\centering
					\def \svgwidth{0.75\columnwidth}
					\input{van_der_poll_5_lin.pdf_tex}
					\caption{Układ zlinearyzowany}
				\end{figure}\noindent
				
				
				
				\begin{figure}[H]
					\centering
					\def \svgwidth{0.75\columnwidth}
					\input{van_der_poll_5_both_close.pdf_tex}
					\caption{TODO}
				\end{figure}\noindent
			
			\subsubsection{Współczynnik a = 0}
				\begin{figure}[H]
					\centering
					\def \svgwidth{0.75\columnwidth}
					\input{van_der_poll_0.pdf_tex}
					\caption{Układ nieliniowy}
				\end{figure}\noindent
			
			
			\subsubsection{Współczynnik a = 0,5}
				\begin{figure}[H]
					\centering
					\def \svgwidth{0.75\columnwidth}
					\input{van_der_poll_0_5.pdf_tex}
					\caption{Układ nieliniowy}
				\end{figure}\noindent
				
				
				\begin{figure}[H]
					\centering
					\def \svgwidth{0.75\columnwidth}
					\input{van_der_poll_0_5_lin.pdf_tex}
					\caption{Układ zlinearyzowany}
				\end{figure}\noindent
				
				
				\begin{figure}[H]
					\centering
					\def \svgwidth{0.75\columnwidth}
					\input{van_der_poll_0_5_both_close.pdf_tex}
					\caption{TODO}
				\end{figure}\noindent
			
		\subsection{Wahadło tłumione}
			\subsubsection{Współczynniki a=1, b=1}
				\begin{figure}[H]
					\centering
					\def \svgwidth{0.75\columnwidth}
					\input{pendulum_1.pdf_tex}
					\caption{Układ nieliniowy}
				\end{figure}\noindent
				
				
				\begin{figure}[H]
					\centering
					\def \svgwidth{0.75\columnwidth}
					\input{pendulum_1_lin.pdf_tex}
					\caption{Układ zlinearyzowany}
				\end{figure}\noindent
				
				
				\begin{figure}[H]
					\centering
					\def \svgwidth{0.75\columnwidth}
					\input{pendulum_1_both_close.pdf_tex}
					\caption{TODO}
				\end{figure}\noindent
			
			\subsubsection{Współczynnik a = 4, b = 1}
				\begin{figure}[H]
					\centering
					\def \svgwidth{0.75\columnwidth}
					\input{pendulum_2.pdf_tex}
					\caption{Układ nieliniowy}
				\end{figure}\noindent
				
				
				\begin{figure}[H]
					\centering
					\def \svgwidth{0.75\columnwidth}
					\input{pendulum_2_lin.pdf_tex}
					\caption{Układ zlinearyzowany}
				\end{figure}\noindent
				
				
				
				\begin{figure}[H]
					\centering
					\def \svgwidth{0.75\columnwidth}
					\input{pendulum_2_both_close.pdf_tex}
					\caption{TODO}
				\end{figure}\noindent
				
			\subsubsection{Współczynnik a = 4, b = 0}
				\begin{figure}[H]
					\centering
					\def \svgwidth{0.75\columnwidth}
					\input{pendulum_3.pdf_tex}
					\caption{Układ nieliniowy}
				\end{figure}\noindent
				
			\subsection{Współczynnik a = 1, b = 3}
				\begin{figure}[H]
					\centering
					\def \svgwidth{0.75\columnwidth}z
					\input{pendulum_4.pdf_tex}
					\caption{Układ nieliniowy}
				\end{figure}\noindent
				
				
				\begin{figure}[H]
					\centering
					\def \svgwidth{0.75\columnwidth}
					\input{pendulum_4_lin.pdf_tex}
					\caption{Układ zlinearyzowany}
				\end{figure}\noindent
				
				
				\begin{figure}[H]
					\centering
					\def \svgwidth{0.75\columnwidth}
					\input{pendulum_4_both_close.pdf_tex}
					\caption{TODO}
				\end{figure}\noindent
			
		\subsection{Układ mechaniczny}
			\subsubsection{Sprężyna twarda}
				\begin{figure}[H]
					\centering
					\def \svgwidth{0.75\columnwidth}
					\input{mechanical_1.pdf_tex}
					\caption{Układ nieliniowy}
				\end{figure}\noindent
				
				
				\begin{figure}[H]
					\centering
					\def \svgwidth{0.75\columnwidth}
					\input{mechanical_1_lin.pdf_tex}
					\caption{Układ zlinearyzowany}
				\end{figure}\noindent
				
				
				\begin{figure}[H]
					\centering
					\def \svgwidth{0.75\columnwidth}
					\input{mechanical_1_both_far.pdf_tex}
				\end{figure}\noindent
				
				
				\begin{figure}[H]
					\centering
					\def \svgwidth{0.75\columnwidth}
					\input{mechanical_1_both_close.pdf_tex}
					\caption{TODO}
				\end{figure}\noindent
				
			\subsubsection{Sprężyna miękka}
				\begin{figure}[H]
					\centering
					\def \svgwidth{0.75\columnwidth}
					\input{mechanical_2.pdf_tex}
					\caption{Układ nieliniowy}
				\end{figure}\noindent
				
				
				\begin{figure}[H]
					\centering
					\def \svgwidth{0.75\columnwidth}
					\input{mechanical_2_lin.pdf_tex}
					\caption{Układ zlinearyzowany}
				\end{figure}\noindent
				
				
				\begin{figure}[H]
					\centering
					\def \svgwidth{0.75\columnwidth}
					\caption{TODO}
				\end{figure}\noindent
				
				
				\begin{figure}[H]
					\centering
					\def \svgwidth{0.75\columnwidth}
					\input{mechanical_2_both_close.pdf_tex}
					\caption{TODO}
				\end{figure}\noindent
		

	\section{Wnioski}
		Traktując regulatory jako filtry można zmieniać czułość regulatora w zależności od częstotliwości podawanego na niego sygnału. Można w ten sposób zapinając sprzężenie zwrotne reagować lepiej na szybkie zmiany sygnału wyjściowego (regulator PD) lub na zmiany powolne (regulator PI) lub na obydwa rodzaje zmian - wykorzystując regulator PID.
		\\ \\
		Filtr pasmowo przepustowy może zostać zrealizowany przy pomocy obiektu drugiego rzędu. Jednak narasta on i opada wtedy  łagodnie na brzegach (20 dB na dekadę). Można te tępo wzrostu i spadku zwiększyć, powiększając bieguny oraz zera transmitancji. Należy jednak być z tym ostrożnym, gdyż zwiększa to rząd transmitancji, a to może być niekorzystne w sytuacji zapinania sprzężenia zwrotnego -- układ może stać się łatwo niestabilny.
		
\end{document}
