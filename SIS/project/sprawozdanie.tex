\documentclass[a4paper, 12pt, titlepage]{article}

\usepackage[utf8]{inputenc}
\usepackage{geometry}
\usepackage{polski}
\usepackage{graphicx}
\usepackage{float}
\usepackage{etoolbox,refcount}
\usepackage{multicol}
\usepackage{fancyhdr}
\usepackage{listings}
\usepackage{amsmath}
\usepackage{tabularx}
\usepackage{svg}

\newgeometry{left=2.5cm, right=2.5cm, bottom=2.5cm, top=2.5cm}

\lstset{
    language=Matlab,
    basicstyle=\ttfamily,
    keepspaces=true,
    frame=single,
    tabsize=4,
    showspaces=false,
    showstringspaces=false
}

\author{Konrad Czaja, Adrian Jałoszewski,\\Dorota Kowalik}
\title{Akwizycja danych z sensorów urządzeń mobilnych}
\date{11 października 2017}


\begin{document}
    \maketitle
	\section{Cel ćwiczenia}
        Celem ćwiczenia jest implementacja algorytmu zliczającego kroki na
        podstawie danych zebranych z czujników urządzenia mobilnego z systemem
        Android. Algorytm ten powinien być odporny na zachowania inne niż chód
        oraz możliwy do zastosowania w ciągłej akwizycji danych.
    \section{Nawiązanie połączenia}

	\section{Ciągła akwizycja danych}
        \lstinputlisting{./scripts/sample.m}
    \section{Wybór miejsca ułożenia smartfonu}
% tutaj może dać jakiś schemat ludzia lub cuś
    \section{Algorytm zliczający kroki}
%        \begin{figure}[H]
%            \centering
%            \def \svgwidth{0.7\columnwidth}
%            \input{todo.tex}
%            \caption{Schemat blokowy algorytmu}
%        \end{figure}\noindent
    \section{Implementacja algorytmu}
        \lstinputlisting{./scripts/sample.m}
    \section{Połączenie algorytmu z ciągłą akwizycją danych}   
        \lstinputlisting{./scripts/sample.m}
    \section{Wyniki testów}
        \subsection{Wyniki dla danych zebranych zawczasu}
        \subsection{Wyniki dla ciągłej akwizycji danych}
    \section{Wnioski i obserwacje}
\end{document}
