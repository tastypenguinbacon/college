\documentclass[a4paper, 12pt, titlepage]{article}

\usepackage[utf8]{inputenc}
\usepackage{geometry}
\usepackage{polski}
\usepackage{graphicx} 
\usepackage{float} 
\usepackage{etoolbox,refcount}
\usepackage{multicol}
\usepackage{fancyhdr}
\usepackage{listings}
\usepackage{amsmath}
\usepackage{tabularx}

\pdfsuppresswarningpagegroup=1
\newgeometry{left=2.5cm, right=2.5cm, bottom=2.5cm, top=2.5cm}

\lstset{
    language=Matlab,
    basicstyle=\ttfamily,
    keepspaces=true,
    frame=single,
    tabsize=4,
    showspaces=false,
    showstringspaces=false,
    extendedchars=true,
    inputencoding=utf8,
    literate={ó}{{\'o}}1 {ę}{{\k{e}}}1 {ł}{{\l{}}}1 {ż}{{\.z}}1
        {ś}{{\'s}}1 {ć}{{\'c}}1 {ą}{{\k{a}}}1 {ź}{{\'z}}1 {ń}{{\'n}}1
}

\author{Adrian Jałoszewski}
\title{Ćwiczenie 3: Detekcja ruchu}
\date{23 października 2017}

\begin{document}
    \maketitle
    \section{Operacja SAD}
        SAD -- sum of absolute differences -- suma wartości bezwzględnych
        z różnicy pikseli dwóch kolejnych ramek sekwencji wideo.
        $$
            d_{\tau}(I_t, I_{t-1}) = \sum_{x, y} |I_t(x,y) - I_{t-1}(x,y)|
        $$
        W przypadku ruchu zmieniają się piksele obrazu, dlatego też zmienia
        się suma modułów ich różnic. Algorytm ten może być uogólniony z modułu
        na normę -- można wtedy traktować poszczególne piksele jako wektory
        w przestrzeni trójwymiarowej. Aby uniezależnić ruch od oświetlenia 
        można zastosować np. transformację RGB na YCbCr i potraktować składowe
        Cb oraz Cr jako wektor dwuwymiarowy.
    \section{Motion Threshold dla SAD oraz parametry bloków Add i Abs}
        Motion Threshold dla operacji SAD powinien wynosić 320 000 - jest to wartość
        powyżej której na obrazie występował ruch.
        \\ \\
        Na wejściu bloku Add znajdują się dwie liczby całkowite 8-bit bez znaku,
        przez co mają zakres od 0 do 255. Po odjęciu skrajnych wartości od siebie
        możemy mieć wynik w zakresie od -255 do 255, co można zapisać na liczbie
        siedmiobitowej ze znakiem, dlatego też powinna być tu ustawiona liczba 
        całkowita ze znakiem o 16 bitach. Modół z niej ma już zasięg tylko od
        0 do 255, więc blok abs powinien przyjmować liczby całkowite 16-bit ze 
        znakiem, a na wyjściu mieć liczby całkowite 8-bit bez znaku.
    \section{Uniezależnienie wyników od zakłóceń i zwiększenie skuteczności detekcji}
        Aby uniezależnić wynik od zakłóceń można wstępnie przeprowadzić na obrazie
        filtrację medianową o stosunkowo małym rozmiarze (3x3 -- 5x5), pozwoli to
        na usunięcie szumów występujących na obrazie. 
        \\ \\        
        Równie dobrym pomysłem jest zastosowanie ważonej średniej ruchomej pomiędzy 
        pikselami ze współczynnikami wykładniczymi. Ze względu na szybki zanik 
        wartości z początku nie wymagałoby to ich odejmowania. Jednak złe dobranie 
        parametrów może tu poskutkować opóźnieniem w pojawieniu się ruchu --
        ruch będzie dopiero obserwowalny po kilku klatkach.
    \section{Parametry $\tau$ oraz Motion Threshold}
        Algorytm MHI działa dobrze dla następujących parametrów:
        \begin{itemize}
            \item[--] $\tau=256$ oraz Motion Threshold = 28
            \item[--] $\tau=128$ oraz Motion Threshold = 40
        \end{itemize}
        Gdzie parametr Motion Threshold jest minimalną różnicą przy której wykrywany
        jest ruch -- zbyt małe różnice nie są traktowane jako ruch. Parametr $\tau$ 
        jest informacją o zanikaniu ruchu -- przez tyle iteracji w przypadku 
        niewykrycia ruchu dla danej komórki jest on wygaszany (odejmowana jest 
        jedynka).
        \\ \\
        Jako uwaga boczna -- jest to dosyć sprytne podejście do algorytmu, które wymaga
        przemyślenia. Zdecydowanie lepiej by było dla tej implementacji MHI jakby zamiast
        mnożenia i dodawania były tam zwykłe instrukcje warunkowe.
    \section{Inny algorytm -- Lucas-Kanade}
        \textbf{Algorytmy przepływu optycznego (optical flow)} -- jest to grupa
        algorytmów służących do wykrywania względnego ruchu pomiędzy 
        obserwatorem i sceną. Są one wykorzystywane do wykrywania ruchu w
        systemach wizyjnych.
        \\ \\
        Metody te mają na celu wyznaczenie rucchu pomiędzy dwiema ramkami obrazu
        w chwilach czasowych $t$ oraz $t + \Delta t$ dla każdego z wokseli
        obrazu trójwymiarowego. Poprzez zastosowanie przekroji może to być 
        wykorzystane do wyznaczenia ruchu na płaszczyźnie.
        \\ \\
        Przypadek dwówymiarowy jest określony przez równanie:
        $$
            I(x + \Delta x, y + \Delta y, t + \Delta t) = I(x, y, t)
        $$
        Oznacza to, że jasność pewnego piksela w czasie w pewnym punkcie po
        jego przesunięciu pozostaje stała (piksel zostal przesunięty w czasie
        $\Delta t$ o wektor $[\Delta x, \Delta y]$, nie tracąc przy tym na
        jasności. Rozwijając te równanie w szereg taylora dla małych 
        przesunięć w czasie mamy:
        $$
            I(x + \Delta x, y + \Delta y, t + \Delta t) = I(x, y, t) +
                \frac{\partial I}{\partial x} \Delta x +
                \frac{\partial I}{\partial y} \Delta y +
                \frac{\partial I}{\partial t} \Delta t +
                R(x, y, t)
        $$
        Gdzie R jest dla małych różnic w czasie pomijalnie małe. Wynika z tego,
        że:
        $$
            \frac{\partial I}{\partial x} \Delta x +
            \frac{\partial I}{\partial y} \Delta y +
            \frac{\partial I}{\partial t} \Delta t = 0
        $$
        Dzieląc przez $\Delta t$
        $$
            \frac{\partial I}{\partial x} \frac{\Delta x}{\Delta t} +
            \frac{\partial I}{\partial y} \frac{\Delta y}{\Delta t} +
            \frac{\partial I}{\partial t} = 0
        $$
        Dla bardzo małych różnic czasu można zastosować przybliżenie:
        $$
            \frac{\partial I}{\partial x} V_x +
            \frac{\partial I}{\partial y} V_y +
            \frac{\partial I}{\partial t} = 0
        $$
        Należy dla każdego piksela wyznaczyć $V_x$ oraz $V_y$. Ze względu na to,
        że równanie te ma dwie niewiadome stosuje się dodatkowe ograniczenia, 
        które mają na celu ujednoznacznić wynik. Np. Metoda Lucasa-Kanade,
        która korzysta z macierzy pseudoodwrotnej.
\end{document}
