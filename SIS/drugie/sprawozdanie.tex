\documentclass[a4paper, 12pt, titlepage]{article}

\usepackage[utf8]{inputenc}
\usepackage{geometry}
\usepackage{polski}
\usepackage{graphicx}
\usepackage{float}
\usepackage{etoolbox,refcount}
\usepackage{multicol}
\usepackage{fancyhdr}
\usepackage{listings}
\usepackage{amsmath}
\usepackage{tabularx}
\usepackage{svg}

\pdfsuppresswarningpagegroup=1
\newgeometry{left=2.5cm, right=2.5cm, bottom=2.5cm, top=2.5cm}

\lstset{
    language=Matlab,
    basicstyle=\ttfamily,
    keepspaces=true,
    frame=single,
    tabsize=4,
    showspaces=false,
    showstringspaces=false,
    extendedchars=true,
    inputencoding=utf8,
    literate={ó}{{\'o}}1 {ę}{{\k{e}}}1 {ł}{{\l{}}}1 {ż}{{\.z}}1 {ś}{{\'s}}1
        {ć}{{\'c}}1 {ą}{{\k{a}}}1 {ź}{{\'z}}1 {ń}{{\'n}}1
}

\author{Adrian Jałoszewski}
\title{Ćwiczenie 1: akwizycja sekwencji video oraz podstawowe operacje na
        obrazach}
\date{16 października 2017}

\begin{document}
    \maketitle
    \section{Czym różni się zespolony sygnał wizji (Composite video) od
        oddzielonego sygnału wizyjnego (S-Video)?}
        Systemy te różnią się sposobem przesyłania sygnału luminancji oraz
        chrominancji. S-Video ma dedykowaną parę przewodów pod każdy z
        sygnałów, a Composite video wykorzystuje do przenoszenia obydwu 
        sygnałów tylko jedną parę. 
        \\ \\ 
        Sygnał przesyłany przy pomocy S-Video jest wyższej jakości ponieważ
        nie ma potrzeby rozdzielania luminancji i chrominancji. Pozwala to
        na uniknięcie szumów, zwiększenie pasma przenoszenia sygnału luminancji
        oraz nie ma problemu zakłóceń sygnału luminancji sygnałem chrominancji
        -- co prowadzi do wyższej jakości obrazu.

    \section{Porównaj standardy nadawania koloru NTSC, PAL i SECAM Używane
        w telewizji. Jaki standard używany jest w Polsce?}
        \subsection{NTSC (National Television System Commitee)}
            Sygnał jest kodowany przy pomocy składowej luminancji 'Y' oraz
            chrominancji, w skład którego wchodzą sygnały różnicowe 'I' oraz
            'Q' związane z kolorem czerwonym i niebieskim.
            \\ \\
            Wadą systemu jest wrażliwość na pasożytnicze przesunięcia fazowe
            sygału chrominancji pojawiające się przy różnych poziomach
            luminancji w przypadku przesyłania sygnałów łączami radiowymi
            na duże odległości.
            \begin{itemize}
                \item[--] 525 linii
                \item[--] 59,94 Hz -- częstotliwość odświeżania
                \item[--] 29,97 klatek na sekundę
            \end{itemize}
        \subsection{PAL (Phase Alternating Line)}
            Udoskonalona modyfikacja systemu NTSC. Fala podnośna przenosi tu 
            też jednocześnie informacje o dwóch składowych barwy -- jest to
            jednak czerwona i niebieska. Co drugą linię obrazu jedna ze
            składowych ma fazę odwracaną o 180 stopni -- automatyczna korekcja
            błędów różnicowych. Dla korekcji poważniejszych błędów została
            zapożyczona z systemu SECAM linia opóźniająca w odbiornikach --
            dodatkowa informacja z poprzedniej linii pozwala na efektywniejszą
            korekcję w połączeniu z odwróceniem fazy.
            \begin{itemize}
                \item[--] 625 linii
                \item[--] 50 Hz -- częstotliwość odświeżania
                \item[--] 25 klatek na sekundę
            \end{itemize}
        \subsection{SECAM (séquentiel couleur à mémoirei)}
            SECAM utrudnia pracę edytorską nad materiałami, gdyż wymaga 
            demodulacji i ponownej modulacji. Dlatego też podczas 
            wielokrotnej edycji w tym formacie następuje utrata danych.
            Dlatego też popularną praktyką jest edycja w innym systemie,
            a następnie, tuż przed wysłaniem zamiana na SECAM.
            \begin{itemize}
                \item[--] 625 linii
                \item[--] 50 Hz -- częstotliwość odświeżania
                \item[--] 25 klatek na sekundę
            \end{itemize}
        W Polsce system PAL D/K został zastąpiony w 2013 roku przez telewizję
        naziemną DVB-T. Występuje jednak dalej w przypadku telewizji kablowej.
    \section{Akwizycja obrazu przy pomocy programu VirtualDub}
        Oznaczenia:
        \begin{itemize}
            \item[] FPS -- liczba klatek na sekundę
            \item[] n -- liczba klatek
            \item[] N -- rozmiar ramki
            \item[] M -- rozmiar pliku
            \item[] t -- czas ramki
            \item[] T -- czas trwania nagrania
            \item[] BR -- bit rate
            \item[] resoution.x/y -- rozdzielczość w poziomie i pionie
        \end{itemize}
        \subsection{Czas akwizycji jednej ramki obrazu}
            $$
                t = \frac{1}{FPS} = \frac{1}{30} \mathrm{s}
            $$
            Jest to wartość zgodna z wartością rzeczywistą
        \subsection{Rozmiar pliku sekwencji video (AVI)}
            Teoretycznie (podobna wartość odczytana jako pojedyncza klatka 
            zapisana):
            $$
                M = n \cdot N = 541 670 \mathrm{kB}
            $$
            W rzeczywistości:
            $$
                M = 525 145 \mathrm{kB}
            $$
        \subsection{Rozmiar pliku jednej ramki obrazu}
            Teoretycznie:
            $$
                N = 3 \cdot resolution.x \cdot resolution.y = 230400 \mathrm{B}
            $$
            W rzeczywistości:
            $$
                N = \frac{BR \cdot T}{n} =  223 371 \mathrm{B}
            $$
        \\ \\
        Różnica wynika tu z zastosowania kompresji w formacie AVI.
    \section{Akwizycja jednej ramki sygnału w programie MATLAB}
\begin{lstlisting}
vid = videoinput('linuxvideo', 1, 'RGB24_320x240');
src = getselectedsource(vid);
vid.FramesPerTrigger = 1;

preview(vid);
stoppreview(vid);
preview(vid);
start(vid);
stoppreview(vid);

obraz1 = getdata(vid);
imshow(obraz1)
\end{lstlisting}
    \section{Parametry symulacji pozwalające dokonać akwizycji 25 ramek}
        \begin{itemize}
            \item[--] Block Sample Time = 1
            \item[--] Stop Time = 24
        \end{itemize}
    \section{Protokół RTSP}
        RTSP -- Real Time Streaming Protocol -- protokół poziomu aplikacji 
        odpowiedzialny za sterowanie dostarczaniem danych czsu rzeczywistego.
        Protokół obsługuje unicast, multicast oraz multicast z wybraniem
        adresu przez klienta. Nie jest zależny od warstwy transportowej,
        jednak musi zachować stan sesji. Jest to protokół stanowy. Zarówno 
        klient jak i serwer mogą wysyłać żądania.
        Udostępnia następujące operacje:
        \begin{itemize}
            \item[--] Pozyskiwanie danych z serwera danych
            \item[--] ,,Zaproszenie serwera danych do sesji''
        \end{itemize}
        Protokół RSTP jest odpowiedzialny za tworzenie i sterowanie
        strumieniami ciągłych danych.
        \\ \\
        Średnia ilość klatek na sekundę dla danych pobieranych z kamery:
        4,87 FPS
\end{document}
