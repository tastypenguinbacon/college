\documentclass[a4paper, 12pt, titlepage]{article}

\usepackage[utf8]{inputenc}
\usepackage{geometry}
\usepackage{polski}
\usepackage{graphicx}
\usepackage{float}
\usepackage{etoolbox,refcount}
\usepackage{multicol}
\usepackage{fancyhdr}
\usepackage{listings}
\usepackage{amsmath}
\usepackage{tabularx}
\usepackage{svg}

\pdfsuppresswarningpagegroup=1
\newgeometry{left=2.5cm, right=2.5cm, bottom=2.5cm, top=2.5cm}

\lstset{
    language=Matlab,
    basicstyle=\ttfamily,
    keepspaces=true,
    frame=single,
    tabsize=4,
    showspaces=false,
    showstringspaces=false,
    extendedchars=true,
    inputencoding=utf8,
    literate={ó}{{\'o}}1 {ę}{{\k{e}}}1 {ł}{{\l{}}}1 {ż}{{\.z}}1
        {ś}{{\'s}}1 {ć}{{\'c}}1 {ą}{{\k{a}}}1 {ź}{{\'z}}1 {ń}{{\'n}}1
}

\author{Adrian Jałoszewski}
\title{Ćwiczenie 2: Detekcja twarzy}
\date{23 października 2017}

\begin{document}
    \maketitle
    \section{Wpływ doboru parametrów na detekcję twarzy}
    \section{Wyznaczanie modelu koloru skóry}
    \section{Operacja zamknięcia w procesie detekcji twarzy}
    \section{Inne metody lokalizacji twarzy na obrazie}
    \section{Rezultaty detekcji twarzy przy pomocy kaskad Haara}
    \section{Metoda detekcji twarzy przy pomocy kaskad Haara}
\end{document}
