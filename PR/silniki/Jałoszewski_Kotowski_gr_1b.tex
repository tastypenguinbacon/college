\documentclass[a4paper, 12pt, titlepage]{article}
\usepackage[utf8]{inputenc}
\usepackage{geometry}
\usepackage{polski}
\usepackage{graphicx}
\usepackage{float}
\usepackage{etoolbox,refcount}
\usepackage{multicol}
\usepackage{fancyhdr}
\pagestyle{fancy}
\title{Modelowanie i symulacja serwomechanizmu liniowego i nieliniowego}
\author{Adrian Jałoszewski, Tomasz Kotowski}
\date{}
\newgeometry{left=2.5cm, right=2.5cm, bottom=2.5cm, top=2.5cm}

\begin{document}
	\maketitle
	\section{Cel ćwiczenia}
		Celem ćwiczenia jest zamodelowanie serwomechanizmu liniowego (regulator PID dla różnych nastaw) oraz nieliniowego (regulator trójpołożeniowy dla różnych wartości strefy martwej i stref histerezy)
	
\end{document}