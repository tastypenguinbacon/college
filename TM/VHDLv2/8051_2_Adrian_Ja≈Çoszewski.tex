\documentclass[a4paper, 12pt, titlepage]{article}
\usepackage[utf8]{inputenc}
\usepackage{geometry}
\usepackage{polski}
\usepackage{graphicx}
\usepackage{float}
\usepackage{etoolbox,refcount}
\usepackage{multicol}
\usepackage{fancyhdr}
\usepackage{listings}
\title{8051 seria druga}
\author{Adrian Jałoszewski}
\date{}
\newgeometry{left=2.5cm, right=2.5cm, bottom=2.5cm, top=2.5cm}


\begin{document}
	\maketitle
	\lstset{frame=single,keepspaces=true} 
	\section{Port szeregowy - wysyłanie ciągu znaków w pętli głównej}
		Inicjalizacja:
\begin{lstlisting}
INIT:	MOV TMOD, #20H
	MOV TL1, #245	; 2400 baud
	MOV TH1, #245
	MOV SCON, #50H
	SETB TR1
\end{lstlisting}
		Podprocedura wysyłająca dane z akumulatora przez port szeregowy.
\begin{lstlisting}
SEND:	MOV SBUF, A
	JNB TI, $
	CLR TI
	RET
\end{lstlisting}
		Dane do wysłania, zakończone zerem:
\begin{lstlisting}
DATA_TO_TRANSFER:	;"Hello world!"	
	DB 72
	DB 101
	DB 108
	DB 108
	DB 111
	DB 32
	DB 87
	DB 111
	DB 114
	DB 108
	DB 100
	DB 33
	DB 0
\end{lstlisting}
		Pętla główna:
\begin{lstlisting}
MAIN:	MOV DPTR, #DATA_TO_TRANSFER
	MOV A, #0
REPEAT:
	PUSH A
	MOVC A, @A+DPTR
	JZ CLEAR_STACK
	ACALL SEND
	POP A
	INC A
	SJMP REPEAT
CLEAR_STACK:
	POP A
	SJMP MAIN
\end{lstlisting}
	\section{Port szeregowy - echo}
		Podprocedura odbierająca dane do akumulatora:
\begin{lstlisting}
RECV:	JNB RI, $
	MOV A, SBUF
	CLR RI
	RET
\end{lstlisting}
		Pętla główna:
\begin{lstlisting}
MAIN:	ACALL RECV
	ACALL SEND
\end{lstlisting}
		Echo selektywne wyświetlające litery od dużego 'A' do dużego 'Z'
\begin{lstlisting}
MAIN:	ACALL RECV
	CJNE A, #0x5a, NOT_Z
	SJMP SEND_LETTER
NOT_Z:	JNC MAIN	; >= 'Z'
	CJNE A, #0x41, NEXT
NEXT:	JNC SEND_LETTER ; >= 'A'
	SJMP MAIN
SEND_LETTER:
	ACALL SEND
	SJMP MAIN
\end{lstlisting}
		Echo selektywne na przerwaniach, w inicjalizacji dodatkowo jest \texttt{MOV IE, \#0x90}, dla włączenia przerwań:
\begin{lstlisting}
HANDLE_SERIAL_INTERRUPT:
	PUSH A
	JNB TI RECEIVED
	CLR TI
	SJMP END_SERIAL_INTERRUPT
RECEIVED:
	MOV A, SBUF
	CLR, RI
	CJNE A, #0x5a, NOT_Z
	SJMP SEND_LETTER
NOT_Z:	JNC END_SERIAL_INTERRUPT	; >= 'Z'
	CJNE A, #0x41, NEXT
NEXT:	JNC SEND_LETTER ; >= 'A'
	SJMP END_SERIAL_INTERRUPT
SEND_LETTER:
	MOV SBUF, A
END_SERIAL_INTERRUPT:
	POP A
	RETI
\end{lstlisting}
	\section{Wyświetlacze - statyczne wyświetlanie}
	Tablica przekodowań na wyświetlacz siedmiosegmentowy:
\begin{lstlisting}
LED_TABLE:	;0...9
     DB 0xc0 
     DB 0xf9
     DB 0xa4
     DB 0xb0
     DB 0x99
     DB 0x92      
     DB 0x82
     DB 0xf8      
     DB 0x80      
     DB 0x90
\end{lstlisting}
	Kod inicjalizujący program - ustawia timer 16 bitowy oraz przerwania od niego, również inicjalizuje poszczególne cyfry.
\begin{lstlisting}
INIT:	MOV	TMOD,	#0x1
	MOV	IE,	#0x82
	MOV	TH0,	#60
	MOV	TL0,	#176
	SETB	TR0
	MOV	DPTR,	#LED_TABLE
	MOV	R0,	#0
	MOV	R1,	#1
	MOV	R2,	#2
	MOV	R3,	#3
	MOV R5, #0
	SJMP $
\end{lstlisting}
	Obsługa przerwania od timera:
\begin{lstlisting}
T0_ISR:    
	PUSH A
	CLR	TR0
	MOV	TH0,    #60
	MOV	TL0,    #176
	SETB	TR0
	CALL	CHANGE_LED
	POP A
	RETI
\end{lstlisting}
\newpage\noindent
Funkcja odświeżająca wyświetlacze, R5 trzyma stan wyświetlacza aktywnego:
\begin{lstlisting}
CHANGE_LED:
	MOV P0, #FF 
	CJNE R5, #3, NOT3
	MOV P1, #8
	MOV A, R0
	MOVC A, @A+DPTR
	MOV	P0, A
	MOV R5 #2
	RET
NOT3:	CJNE R5, #2, NOT2
	MOV P1, #8
	MOV A, R1
	MOVC A, @A+DPTR
	MOV	P0, A
	MOV R5 #1
	RET
NOT2:	CJNE R5, #1, NOT1
	MOV P1, #8
	MOV A, R2
	MOVC A, @A+DPTR
	MOV	P0, A
	MOV R5 #0
	RET
NOT1:	MOV R5, #3
	MOV P1, #8
	MOV A, R3
	MOVC A, @A+DPTR
	MOV	P0, A
	MOV R5 #3
	RET
\end{lstlisting}
\newpage\noindent
\section{Licznik}
Kod inicjalizujący program - ustawia timer 16 bitowy oraz przerwania od niego, również inicjalizuje poszczególne cyfry, również inicjalizuje rejestr R5 wartością 201, gdyż mając częstotliwość odświeżania 50Hz przełączamy między czterema wyświetlaczami co 5ms, więc po upływie 200 iteracji mamy upływ sekundy:
\begin{lstlisting}
INIT:	MOV	TMOD,	#0x1
	MOV	IE,	#0x82
	MOV	TH0,	#60
	MOV	TL0,	#176
	SETB	TR0
	MOV	DPTR,	#LED_TABLE
	MOV	R0,	#0
	MOV	R1,	#0
	MOV	R2,	#0
	MOV	R3,	#0
	MOV	R4, #201
	MOV R5, #0
	SJMP $

\end{lstlisting}
Obsługa przerwania od timera:
\begin{lstlisting}
T0_ISR:    
	PUSH A
	CLR	TR0
	MOV	TH0,    #60
	MOV	TL0,    #176
	SETB	TR0
	DJNZ R4, NEXT
	MOV R4, #201
	CALL UPDATE_COUNTER
	NEXT:	CALL	CHANGE_LED
	POP A
RETI
\end{lstlisting}
Funkcja uaktualiniająca wartości R0, R1, R2, R3, na podstawie wartości R6 zwiększa kolejną wartość o 1 lub nie.
\begin{lstlisting}
UPDATE_COUNTER:
	MOV R6, #1
	MOV A, R0
	ACALL GET_MODULO_TEN
	MOV R0, A
	MOV A, R1
	ACALL GET_MODULO_TEN
	MOV R1, A
	MOV A, R2
	ACALL GET_MODULO_TEN
	MOV R2, A
	MOV A, R3
	ACALL GET_MODULO_TEN
	MOV R3, A
	RET
\end{lstlisting}
Funkcja odpowiedzialna za zmianę wartości, zapisuje do R6 1, jeżeli nastąpiło przepełnienie:
\begin{lstlisting}
GET_MODULO_TEN:
	CJNE R6, #1, RETURN_MODULO_TEN
	MOV R6, #0
	INC A
	CJNE A, #10, RETURN_MODULO_TEN
	MOV A, #0
	MOV R6, #1
RETURN_MODULO_TEN:
	RET
\end{lstlisting}
\section{Zegarek}
	Aby zrobić z tego zegarek należy zamienić co drugie wywołanie funkcji GET\_MODULO\_TEN na GET\_MODULO\_SIX
\begin{lstlisting}
UPDATE_COUNTER:
	MOV R6, #1
	MOV A, R0
	ACALL GET_MODULO_TEN
	MOV R0, A
	MOV A, R1
	ACALL GET_MODULO_SIX
	MOV R1, A
	MOV A, R2
	ACALL GET_MODULO_TEN
	MOV R2, A
	MOV A, R3
	ACALL GET_MODULO_SIX
	MOV R3, A
	RET
\end{lstlisting}
Funkcja GET\_MODULO\_SIX:
\begin{lstlisting}
GET_MODULO_SIX:
	CJNE R6, #1, RETURN_MODULO_TEN
	MOV R6, #0
	INC A
	CJNE A, #6, RETURN_MODULO_SIX
	MOV A, #0
	MOV R6, #1
RETURN_MODULO_SIX:
	RET
\end{lstlisting}
\end{document}