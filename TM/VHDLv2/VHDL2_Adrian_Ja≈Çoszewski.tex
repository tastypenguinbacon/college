\documentclass[a4paper, 12pt, titlepage]{article}
\usepackage[utf8]{inputenc}
\usepackage{geometry}
\usepackage{polski}
\usepackage{graphicx}
\usepackage{float}
\usepackage{etoolbox,refcount}
\usepackage{multicol}
\usepackage{fancyhdr}
\usepackage{listings}
\title{VHDL seria druga}
\author{Adrian Jałoszewski}
\date{}
\newgeometry{left=2.5cm, right=2.5cm, bottom=2.5cm, top=2.5cm}

\begin{document}
	\maketitle
	\section{Blok entity}
\lstset{language=VHDL, tabsize=4,frame=single,keepspaces=true} 
\begin{lstlisting}[frame=single, keepspaces=true] 
entity dynamiczny is
	Port (
		CLK : in STD_LOGIC;
		DIS : out STD_LOGIC_VECTOR(6 downto 0);
		AN : out STD_LOGIC_VECTOR(7 downto 0)
	);
end dynamiczny;
\end{lstlisting}
	\section{Funkcje pomocnicze}
		Funkcja przyjmująca liczbę jako argument i zwracająca zakodowaną cyfrę z wyświetlacza.
\begin{lstlisting}[frame=single,keepspaces=true]
function selectNumberToDisplay ( x : integer range 0 to 9) 
	return STD_LOGIC_VECTOR is
	variable temp : std_logic_vector(7 downto 0);
begin
	case x is
		when 0 => temp:= not x"3f";
		when 1 => temp:= not x"06";
		when 2 => temp:= not x"5b";
		when 3 => temp:= not x"4f";
		when 4 => temp:= not x"66";
		when 5 => temp:= not x"6d";
		when 6 => temp:= not x"7d";
		when 7 => temp:= not x"07";
		when 8 => temp:= not x"7f";
		when 9 => temp:= not x"6f";
		when others => temp:= not x"3f";
	end case;
	return temp(6 downto 0);
end selectNumberToDisplay;
\end{lstlisting}
\newpage \noindent
		Funkcja przyjmująca jako argument numer wyświetlacze i zwracająca wartość zapalającą go:
\begin{lstlisting}[frame=single,keepspaces=true]
function selectDisplay(x : integer range 1 to 8) 
	return std_logic_vector is
begin
	case x is
		when 1 => return not x"01";
		when 2 => return not x"02";
		when 3 => return not x"04";
		when 4 => return not x"08";
		when 5 => return not x"10";
		when 6 => return not x"20";
		when 7 => return not x"40";
		when 8 => return not x"80";
		when others => return not x"00";
	end case;
end selectDisplay;
\end{lstlisting}
	\section{Proces odpowiadający za wyświetlanie ośmiu cyfr na wyświetlaczu}
\begin{lstlisting}[frame=single,keepspaces=true]
process(CLK)
	variable CNT : integer :=0;
	variable oneDisplayTime : integer := 250000;
	variable selected_display : integer := 1;
begin 	 
	if rising_edge(CLK) then
		if CNT < oneDisplayTime then
			CNT:=CNT+1;
		else
			selected_display := selected_display + 1;
			if (selected_display > 8) then
				selected_display := 1;
			end if;
			CNT:=0;
		end if;
		DIS <= selectNumberToDisplay(selected_display);
		AN <= selectDisplay(selected_display);
	end if;
end process;   
	\section{title}
\end{lstlisting}
	\section{Wyświetlanie liczb podanych na przełącznikach na poszczególnych wyświetlaczach}
\begin{lstlisting}[frame=single,keepspaces=true]
process(CLK)
	variable CNT : integer :=0;
	variable oneDisplayTime : integer := 500000;
	variable selected_display : integer := 1;
	variable first : integer;
	variable second : integer;
	variable third : integer;
	variable fourth : integer;
begin 	 
	if rising_edge(CLK) then
		first := to_integer(unsigned(sw(3 downto 0)));
		second := to_integer(unsigned(sw(7 downto 4)));
		third := to_integer(unsigned(sw(11 downto 8)));
		fourth := to_integer(unsigned(sw(15 downto 12)));
		
		if CNT < oneDisplayTime then
			CNT:=CNT+1;
		else
			selected_display := selected_display + 1;
			if (selected_display > 4) then
				selected_display := 1;
			end if;	
			CNT:=0;
		end if;
		case selected_display is
			when 1 =>
				DIS <= selectNumberToDisplay(first);
				AN <= selectDisplay(selected_display);
			when 2 =>
				DIS <= selectNumberToDisplay(second);
				AN <= selectDisplay(selected_display);
			when 3 =>
				DIS <= selectNumberToDisplay(third);
				AN <= selectDisplay(selected_display);
			when 4 =>
				DIS <= selectNumberToDisplay(fourth);
				AN <= selectDisplay(selected_display);
			when others =>
				DIS <= selectNumberToDisplay(0);
				AN <= selectDisplay(selected_display);
		end case;
	end if;
end process;   
\end{lstlisting}
	\section{Licznik wyświetlający na czterech wyświetlaczach, inkrementujący co sekundę}
	Dodatkowe sygnały dodane na potrzeby kodu:
\begin{lstlisting}
	signal one_second_counter : integer := 0;
	type INT_ARRAY is array (integer range <>) of integer;
	signal display_values : INT_ARRAY(4 downto 1);
\end{lstlisting}
	Proces inkrementujący licznik wewnątrz co sekundę:
\begin{lstlisting}
every_second:process(CLK)
	variable counter : integer := 0;
	constant one_second : integer := 100_000_000;
begin
	if rising_edge(CLK) then
		counter := counter + 1;
		if counter >= one_second then
			counter := 0;
			one_second_counter <= one_second_counter + 1;
		end if;
	end if; 
end process;
\end{lstlisting}
	Proces wyświetlający poszczególne cyfry:
\begin{lstlisting}
display:process(CLK)
	variable CNT : integer :=0;
	constant oneDisplayTime : integer := 500000;
	variable selected_display : integer := 1;
begin      
	if rising_edge(CLK) then
		if CNT < oneDisplayTime then
			CNT:=CNT+1;
		else 
			selected_display := selected_display + 1;
			if (selected_display > 4) then
				selected_display := 1;
			end if; 
			CNT:=0;
		end if;
	DIS <= decode_digit_to_ssd(display_values(selected_display));
		AN <=  select_display(selected_display);
	end if;
end process;
\end{lstlisting}
	\newpage\noindent
	Przypisywanie wartości do wyświetlania odbywa się w części asynchronicznej programu:
\begin{lstlisting}
	display_values(1) <= one_second_counter mod 10;
	display_values(2) <= one_second_counter / 10 mod 10;
	display_values(3) <= one_second_counter / 100 mod 10;
	display_values(4) <= one_second_counter / 1000 mod 10;
\end{lstlisting}
\section{Zegarek ze wskaźnikiem minut i sekund}
	Aby z powyższego kodu otrzymać czasomierz należy zmienić wartości przypisywanie wartości do wyświetlania na:
\begin{lstlisting}
	display_values(1) <= one_second_counter mod 10;
	display_values(2) <= one_second_counter / 10 mod 6;
	display_values(3) <= one_second_counter / 60 mod 10;
	display_values(4) <= one_second_counter / 600 mod 6;
\end{lstlisting}

\end{document}