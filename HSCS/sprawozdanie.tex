\documentclass[a4paper, 12pt]{article}

\usepackage[utf8]{inputenc}
\usepackage{geometry}
\usepackage{polski}
\usepackage{graphicx}
\usepackage{float}
\usepackage{etoolbox,refcount}
\usepackage{multicol}
\usepackage{fancyhdr}
\usepackage{listings}
\usepackage{amsmath}
\usepackage{tabularx}
\usepackage{svg}

\pdfsuppresswarningpagegroup=1
\newgeometry{left=2.5cm, right=2.5cm, bottom=2.5cm, top=2.5cm}

\lstset{
    language=Fortran,
    basicstyle=\ttfamily,
    keepspaces=true,
    frame=single,
    tabsize=4,
    showspaces=false,
    showstringspaces=false,
    extendedchars=true,
    inputencoding=utf8,
}

\author{Adrian Jałoszewski}
\title{The implementation of ''walking one'' algorithm at SIEMENS 1200}
\date{}

\begin{document}
    \maketitle
    \section{Introduction}
        \subsection{Uses of the algorithm}
            The ''walking one'' algorithm is one of the most well known
            algorithms used for testing the independence of operations
            performed by changing the state of consecutive bits.
            
            It is used for checking whether there is coupling between 
            independent bits (for example -- turning the sixth bit has effect on
            the fifth one).
        \subsection{Algorithm description}
            
            It is implemented using a bitfield consisting of zeros and a single 
            one as the algorithms output.  Every step of the algorithm is 
            basically a cyclic bitshift by one bit in a given direction. The
            shift direction should be specified during the design phase and be
            constant throughout the algorithms runtime.
    \section{Implementation}
\end{document}
